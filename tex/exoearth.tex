\documentclass[preprint,12pt]{aastex}

\def\mearth{{\rm\,M_\oplus}}
\def\rearth{{\rm\,R_\oplus}}
\def\msun{{\rm\,M_\odot}}
\def\rsun{{\rm\,R_\odot}}
\def\lsun{{\rm\,L_\odot}}
\def\gsim{~\rlap{$>$}{\lower 1.0ex\hbox{$\sim$}}}
\def\lsim{~\rlap{$<$}{\lower 1.0ex\hbox{$\sim$}}}
\def\etal{{\it et al.\thinspace}}
\def\wpmsq{W m$^{-2}$}
\def\etal{{\it et al.\thinspace}}
\def\eg{{\it e.g.\ }}
\def\etc{{\it etc.\ }}
\def\ie{{\it i.e.\ }}
\def\cf{{\it c.f.\ }}

\def\tess{{\it TESS}}
\def\kepler{{\it Kepler}}
\def\jwst{{\it JWST}}

\def\vplanet{\texttt{VPLANET}}
\def\atmesc{\texttt{ATMESC}}
\def\distorb{\texttt{DISTORB}}
\def\distrot{\texttt{DISTROT}}
\def\eqtide{\texttt{EQTIDE}}
\def\poise{\texttt{POISE}}
\def\radheat{\texttt{RADHEAT}}
\def\thermint{\texttt{THERMINT}}
\def\stellar{\texttt{STELLAR}}

\bibliographystyle{apj}
\usepackage{multicol}

\begin{document}

\title{A Simple Model for Earth-like Exoplanets}
\author{Rory Barnes\altaffilmark{1,2,3} et al.}
\altaffiltext{1}{Astronomy Department, University of Washington, Box 951580, Seattle, WA 98195}
\altaffiltext{2}{NASA Astrobiology Institute -- Virtual Planetary Laboratory Lead Team, USA}
\altaffiltext{3}{E-mail: rory@astro.washington.edu}

\begin{abstract}
\end{abstract}

\section{Introduction\label{sec:intro}}

In the last several years, astronomers have begun to discover
Earth-sized planets orbiting their host stars such that they receive
similar levels of incident radiation as Earth and hence may be abodes
for life
\citep[\eg][]{AngladaEscude12,Borucki13,Torres15,Barnes15_hite,Morton16}. While
these exoplanets could be rocky and inside their host stars' habitable
zones (HZs) \citep{Hart79,Kasting93,Selsis07,Kopparapu13}, planetary
habitability is a complicated process that depends on numerous
phenomena with timescales ranging from a fraction of a second to eons,
and lengthscales from the atomic to the galactic. The relevant
processes include stellar evolution, orbital oscillations, internal
evolution, and atmospheric effects. The breadth and dynamic range of
the problem of planetary habitability has heretofore prevented a
comprehensive model to address all the relevant aspects, especially
for exoplanets. Below, we describe a quantitative physical model to
explore planetary systems in their entirety, and thus produce a
comprehensive picture of an exoplanet's potential to support life.

Since their initial discovery over 20 years ago
\citep{MayorQueloz95}, the gold standard for assessing
an exoplanet's habitability has been its position in the host star's
HZ as defined in \cite{Kasting93}. Their model relies on a
one-dimensional atmospheric model that calculates surface temperature
as a function of a planet's semi-major axis $a$, the host star's
luminosity $L$ and effective temperature $T_{eff}$, and with the
assumption that atmospheric convection and photochemistry are very
similar to Earth's. The reduction of planetary habitability to this
model implies the energy budget in a planet's troposphere is the
paramount factor in a planet's habitability. This decision may be
reasonable, but it is also well-understood that other factors could
supersede the irradiation requirements
\citep[\eg][]{Joshi97,Segura03,Barnes08,Segura10,Spiegel10,LugerBarnes15,DriscollBarnes15}; a
planet's position in the HZ is just the first of many requirements to
be met.

The goal of this study is to develop a quantitative model that
considers a broader range of phenomena. We will draw on simple, but
proven models from the fields of astrophysics, planetary science,
atmospheric science and Earth science. Bringing these models together
allows a treatment of stellar, orbital, rotational, atmospheric and
internal evolution of planetary systems consisting of 1 star and
multiple planets in a wide array of architectures. While this
treatment is far more general than previous studies, it is still
incomplete. Nonetheless, these models can be used to provide a first
order picture of the evolution and current state of planetary systems
that contain one or more potentially habitable planets.

This paper is organized as follows. In $S$~\ref{sec:models} we
describe the 8 fundamental models that form the basis of our
interpretations of planetary habitability. We use the stellar
evolution models of \citep{Baraffe15} ($\S$~\ref{sec:models}.1), a 4th
order secular model of orbital evolution ($\S$~\ref{sec:models}.2), a
semi-analytic model for the obliquity and precession angle due to
orbital evolution and the stellar torque ($\S$~\ref{sec:models}.3),
radiogenic heating ($\S$~\ref{sec:models}.4), thermal evolution of
terrestrial planet interiors ($\S$~\ref{sec:models}.5), atmospheric
escape due to XUV radiation ($\S$~\ref{sec:models}.6), tidal effects
($\S$~\ref{sec:models}.7), and an energy balance model with a
realistic treatment of ice sheet growth and retreat
($\S$~\ref{sec:models}.8). $\S$~\ref{sec:models} concludes with a
description of how we couple all these individual models into a
software package called \vplanet. In $\S$~\ref{sec:trappist} we apply
this coupled model to the TRAPPIST-1 system, which consists of
0.08~$\msun$ host star and at 3 $\sim$Earth-sized planets near the
classical HZ \citep{Gillon16}. In $\S$~\ref{sec:discussion} we discuss
the value of the coupled model and how it may be used to prioritize
target for life-detection observations. We conclude in
$\S$~\ref{sec:conclusions}.

\section{Models\label{sec:models}}

\subsection{Stellar Evolution: \stellar}
% Rodrigo

% Fig 1: Luminosity, Radius, Temperature, RotPer evolution
% Fig 2: XUV Evolution

\subsection{Orbital Evolution: \distorb}
% Russell

% Fig 3: Jup+Sat, ecc, varpi, incl, Omega
% Fig 4: Earth ecc from all 8 planets

\subsection{Rotational Evolution from Orbits and the Stellar Torque: \distrot}
% Russell

% Fig 5: Martian obliquity

\subsection{Radiogenic Heating: \radheat}
% Peter

% Fig 6: Earth's mantle and core

\subsection{Geophyiscal Evolution: \thermint}
% Peter

% Fig 7: Earth temperatures, heat flows, plate speeds, outgassing
% Fig 8: Inner core growth, magnetic fields, etc.

\subsection{Atmospheric Escape: \atmesc}
% Rodrigo

% Fig 9: Venus, Kepler-36 b+c (Lopez & Fortney)
% Fig 10: O2 build-up

\subsection{Tidal Evolution: \eqtide}
% Rory

\subsubsection{Equilibrium Tides}

The tidal model we use is commonly called the ``equilibrium tide''
model and was first conceived by George Darwin, grandson of Charles
\citep{Darwin1880}. This model assumes the gravitational potential of
the tide raiser on an unperturbed spherical surface can be expressed
as the sum of Legendre polynomials (\ie surface waves) and that the
elongated equilibrium shape of the perturbed body is slightly
misaligned with respect to the line that connects the two centers of
mass. This misalignment is due to dissipative processes within the
deformed body and leads to a secular evolution of the orbit as well as
the spin angular momenta of the two bodies. Furthermore, the bodies
are assumed to respond to the time-varying tidal potential as though
they are damped, driven harmonic oscillators, a well-studied
system. As described below, this approach leads to a set of 6 coupled,
non-linear differential equations, but note that the model is linear
in the sense that there is no coupling between the surface waves which
sum to the equilibrium shape. A substantial body of research is
devoted to tidal theory
\cite[\eg][]{Darwin1880,GoldreichSoter66,Hut81,FerrazMello08,Wisdom08,EfroimskyWilliams09,Leconte10}, and the reader is
referred to these studies for a more complete description of the
derivations and nuances of equlibrium tide theory. For this investigation, I
will use the models and nomenclature of \cite{Heller11}, which are
presented below.

Equilibrium tide models have the advantage of being semi-analytic, and hence can be
used to explore parameter space quickly. They reduce the tidal effects
to a single parameter, which is valuable in systems for which very
little compositional and structural information is known, \eg
exoplanets. However, they suffer from self-inconsistencies. A
rotating, tidally deformed body does not in fact possess multiple
rotating tidal waves that create the non-spherical equilibrium shape
of a body. The properties of the tidal bulge are due to rigidity,
viscosity, structure and frequencies. Equilibrium tide models are not much more than
toy models for tidal evolution -- self-consistent models would require
three dimensions and include the rheology of the interior and, for
ocean-bearing worlds, a 3-dimensional model of currents, ocean floor
topography and maps of continental margins. For exoplanets, such a
complicated model is not available, nor is it necessarily warranted
given the dearth of observational constraints.

The equilibrium tide frameworks permits fundamentally different assumptions
regarding the lag between the passage of the perturber and the passage
of the tidal bulge. This ambiguity has produced two well-developed
models that have reasonably reproduced observations in our Solar
System, but which can diverge significantly when applied to
configurations with different properties. One model assumes that the
lag is a constant in phase and is independent of frequency. In other
words, regardless of orbital and rotational frequencies, the angle
between the perturber and the tidal bulge remains constant. Following
\cite{Greenberg09} we will refer to this version as the
``constant-phase-lag" or CPL model. At first glance, this may seem to
be the best choice, given the body is expected to behave like a
harmonic oscillator: In order for the tidal waves to be linearly
summed, the damping must be independent of frequency. However, for
eccentric orbits, it may not be possible for the phase lag to remain
constant as the orbital frequency changes in accordance with Kepler's
2nd Law \citep{ToumaWisdom94,EfroimskyMakarov13}. This has led
numerous researchers to reject the CPL model, despite its relative
success at reproducing features in the Solar System
\citep[\eg][]{MacDonald64,Hut81,GoldreichSoter66,Peale79}, as well as
the tidal circularization of close-in exoplanets \citep{Jackson08a}.

The second possibility for the lag is that the time interval between
the perturber's passage and the tidal bulge is constant. In this case,
as frequencies change, the angle between the bulge and the perturber
changes. While this model has also been used extensively \citep{Leconte10,Bolmont12,Barnes13,Barnes16}, we will not use it here.

In terms of planetary rotation rate, many of the timescales are set by
masses, radii, and semi-major axes. For typical main sequence stars,
and Mars- to Neptune-sized planets in the classic HZ of
\cite{Kasting93}, the timescales range from millions to trillions of
years, with the shortest timescales occurring for the largest planets
orbiting closest to the smallest stars.

The equilibrium tide framework is limited to two bodies and consists
of 6 independent parameters: the semi-major axis $a$, the eccentricity
$e$, the two rotation rates $\Omega_j$, and two obliquities $\psi_j$,
where $j=1,2$ corresponds to one of the bodies. If the gravitational
gradient across a freely rotating body induces sufficient strain on
that body's interior to force movement, then frictional heating is
inevitable. The energy for this heating comes at the expense of the
orbit and/or rotational frequency, and hence tidal friction decreases
the semi-major axis and rotation period. The friction will also
prevent the elongation of the body to align exactly with the
perturber. With an asymmetry introduced, torques arise and open
pathways for angular momentum exchange. There are three reservoirs of
angular momentum: the orbit and the two rotations. Equilibrium tide
models assume orbit-averaged shapes when calculating torques, which is
a good approximation for the long-term evolution of the system. The
redistribution of angular momentum depends on the tidal power, and the
heights and positions of the tidal bulges relative to the line
connecting the two centers of mass. The equilibrium tide model can
therefore be seen as the angular momentum evolution of two bodies on
an orbit that is losing energy.

The tidal power and bulge properties depend on the composition and
microphysics of planetary and stellar interiors, which are very
difficult to measure in our Solar System, let alone in an external
planetary system. In equilibrium tide theory, the coupling between energy
dissipation and the tidal bulge is therefore a central feature, and
are scaled by two parameters, the Love number of degree 2, $k_2$, and
a parameter that represents the lag between the line connecting the
two centers of mass and the direction of the tidal bulge. In the CPL
model, this parameter is the ``tidal quality factor" $Q$.

Although energy dissipation results in semi-major axis decay, angular
momentum exchange can lead to semi-major axis growth. This can occur
if enough rotational angular momentum can be transferred to the orbit
to overcome the decay due to tidal heating. Earth and the Moon are in
this configuration now.

For planets on eccentric orbits, the rotation rates of the bodies
compared to their instantaneous angular velocity at pericenter
determines how $e$ changes. If the rotational bulge leads the
perturber, there will be a net force in the direction of the orbit at
pericenter, which acts to accelerate the perturber and increase its
eccentricity. If the bulge lags the perturber, the force retards the
orbital velocity and the orbit circularizes. This latter case is
typical of most close-in planets, and so the orbits circularize
\citep{Rasio96,Jackson08a}.

\subsubsection{The Constant Phase Lag Model}

In the CPL model of tidal evolution, the angle between the line connecting the
centers of mass and the tidal bulge is constant. This approach is commonly
utilized in planetary studies \citep[e.g.][]{GoldreichSoter66,Greenberg09} and the
evolution is described by the following equations:

\begin{equation}\label{eq:e_cpl}
  \frac{\mathrm{d}e}{\mathrm{d}t} \ = \ - \frac{ae}{8 G M_* M_p}
  \sum\limits_{i = 1}^2Z'_i \Bigg(2\varepsilon_{0,i} - \frac{49}{2}\varepsilon_{1,i} + \frac{1}{2}\varepsilon_{2,i} + 3\varepsilon_{5,i}\Bigg)
\end{equation}

\begin{equation}\label{eq:a_cpl}
  \frac{\mathrm{d}a}{\mathrm{d}t} \ = \ \frac{a^2}{4 G M_* M_p}
  \sum\limits_{i = 1}^2 Z'_i  \ {\Bigg(} 4\varepsilon_{0,i} + e^2{\Big [} -20\varepsilon_{0,i} + \frac{1
47}{2}\varepsilon_{1,i} + \nonumber \frac{1}{2}\varepsilon_{2,i} - 3\varepsilon_{5,i} {\Big ]} -4\sin^2(\psi_i){\Big [}\varepsilon_{0,i}-\varepsilon_{8,i}{\Big ]}{\Bigg )} 
\end{equation}

\begin{equation}\label{eq:o_cpl}
  \frac{\mathrm{d}\Omega_i}{\mathrm{d}t} \ = \ - \frac{Z'_i}{8 M_i r_{\mathrm{g},i}
^2 R_i^2 n} {\Bigg (}4\varepsilon_{0,i} + e^2{\Big [} -20\varepsilon_{0,i} + 49\varepsilon_{1,i} + \varepsilon_{2,i} {\Big ]} + \nonumber \ 2\sin^2(\psi_i) {\Big [} -
2\varepsilon_{0,i} + \varepsilon_{8,i} + \varepsilon_{9,i} {\Big ]} {\Bigg )} \\  
\end{equation}

\begin{equation}\label{eq:psi_cpl}
  \frac{\mathrm{d}\psi_i}{\mathrm{d}t} \ = \ \frac{Z'_i \sin(\psi_i)}{4 M_i r_{\mathrm{g},i}^2 R_i^2 n \Omega_i} {\Bigg (} {\Big [} 1-\xi_i {\Big ]}\varepsilon_{0,i} 
+ {\Big [} 1+\xi_i {\Big ]}{\Big \{}\varepsilon_{8,i}-\varepsilon_{9,i}{\Big \}} {\Bigg)} \ ,
\end{equation}

\noindent where $e$ is eccentricity, $t$ is time, $a$ is semi-major axis, $G$ is
Newton's gravitational constant, $M_i$ are the two masses,
$R_i$ are the two radii, $\Omega_i$ are the rotational
frequencies, $\psi_i$ are the obliquities, and $n$ is the mean motion. The
quantity $Z'_i$ is

\begin{equation}\label{eq:Zp}
Z'_i \equiv 3 G^2 k_{2,i} M_j^2 (M_i+M_j) \frac{R_i^5}{a^9} \ \frac{1}{n Q_i} \ ,
\end{equation}

\noindent where $k_{2,i}$ are the Love numbers of order 2, and $Q_i$ are the ``tidal quality factors.'' The parameter $\xi_i$ is

\begin{equation}\label{eq:chi}
\xi_i \equiv \frac{r_{\mathrm{g},i}^2 R_i^2 \Omega_i a n }{ G M_j},
\end{equation}

\noindent where $i$ and $j$ refer to the two bodies, and $r_g$ is the ``radius of gyration,'' \ie the moment of inertia is $M(r_gR)^2$. The signs of the phase lags are

\begin{equation}\label{eq:epsilon}
\begin{array}{l}
\varepsilon_{0,i} = \textrm{sgn}(2 \Omega_i - 2 n)\\
\varepsilon_{1,i} = \textrm{sgn}(2 \Omega_i - 3 n)\\
\varepsilon_{2,i} = \textrm{sgn}(2 \Omega_i - n)\\
\varepsilon_{5,i} = \textrm{sgn}(n)\\
\varepsilon_{8,i} = \textrm{sgn}(\Omega_i - 2 n)\\
\varepsilon_{9,i} = \textrm{sgn}(\Omega_i) \ ,\\
\end{array}
\end{equation}

\noindent with sgn($x$) the sign of any physical quantity $x$, \ie
sgn($x)~=~+1, -1$ or 0.

\cite{Goldreich66} suggested that the equilibrium rotation period, \ie the rotation period of a ``tidally locked" world, for both bodies is
\begin{equation}\label{eq:p_eq_cpl}
P_{eq} = \frac{P}{1 + 9.5e^2}.
\end{equation}
\cite{MurrayDermott99} present a derivation of this expression, which
assumes the rotation rate may take a continuum of values. However, the
CPL model described above only permits 4 ``tidal waves'', and hence does
not allow this continuum, only the 3:2 and 1:1 spin orbit resonances. \cite{Barnes13} suggested
the CPL model should be implemented differently depending on the problem. When
modeling the evolution of a system, one should use the discrete spin
values for self-consistency, \ie as an initial condition, or if forcing the spin to remain tide-locked. However, if calculating the equilibrium
spin period separately, the continuous value of Eq.~(\ref{eq:p_eq_cpl})
should be used. we refer to these rotational states as ``discrete'' and
``continuous'' and will use the former throughout this study.

% Fig 11: Earth-Moon orbital history
% Fig 12: Tidal heating of Io

\subsection{Climate: \poise}
% Russell

% Fig 13: Modern Earth

\subsection{The Coupled Model: \vplanet}

% Fig 14: Milankovitch (Russell)

\subsubsection{Evolution of Tight Binaries}
% Fig 15: Tight binaries (David)
A classic study of the evolution of tight binary stars by \citet{ZahnBouchet89} finds that binaries with an orbital period of less than about 8 days tidally
circularize with the majority of the orbital circularization occurring before the stars reach the Zero Age Main Sequence.  Here we reproduce that 
result by coupling the stellar evolution models of \citet{Baraffe15} incorporated in the \stellar \ module and the equilibrium tide CPL model via \eqtide.  See Section \ref{sec:models} for a more in-depth discussion of the models.

Adopting the initial conditions used in \citet{ZahnBouchet89}, we model an equal mass $1 \msun$ binary with initial orbital eccentricity $e = 0.3$ and an orbital 
period of 5 days.  The initial stellar rotation rate to mean motion ratio is set at $\Omega/n = 3$ in line with the estimates \citet{ZahnBouchet89} who 
assumed conservation of angular momentum during the star's accretion phase.  Both stellar radii of gyration are fixed at $r_g = 0.5$.  For our tidal model, 
we set $Q = 10^5$ and $k_2 = 0.5$, both reasonable values for stars given the wide range of assumed values in the literature \citep{Barnes13}.

The results of the simulation are depicted in Figure \ref{fig:zahn89}.  Qualitatively, our model is in good agreement with the results of \citet{ZahnBouchet89}
shown in their Figure 1.  After an initial increase in orbital eccentricity, the binary circularizes within the first $10^6$ years before the ZAMS in line with previous 
findings.  The transition between increasing to decreasing eccentricity occurs when $e = \sqrt{1/19}$ at the $\Omega/n = 1.5$ transition as expected from the 
CPL model.  The orbital period peaks around $10^4$ years before declining to a minimum and rising again to a 
larger period than the initial value.  One difference between the two results is that \citet{ZahnBouchet89} find an increase in $\Omega/n$ from unity to 
over 2 starting near $10^6$ years before the stars tidally lock again after about $10^9$ years while in our model the stars remain tidally locked after $10^5$ 
years.  This discrepancy can be explained by the different implementations of the two tidal models (see \citet{Zahn89}) and that \citet{ZahnBouchet89} allowed 
$r_g$ to vary according to a simple interpolated formula while we held our value fixed.

\begin{figure} 
\figurenum{15} 
\plotone{zahn89.pdf} 
\caption{Coupled stellar and tidal evolution of a solar twin binary from the premain sequence onward calculated in \vplanet \ using the \eqtide \ and \stellar \ 
modules.  Binary eccentricity evolution is given by the red solid line, the orbital period by the purple dashed line, and the ratio of stellar rotation rate to binary
mean motion ($\Omega/n$) evolution is given by the blue dot dashed curve.
The binary's evolution qualitatively agrees with that of an identical system presented in Figure 1 of \citet{ZahnBouchet89}.} 
\label{fig:zahn89}
\end{figure}


% Fig 16: Tidal heating interiors (Peter)
% Fig 17: Cassini States (Russell)

\section{Application to the TRAPPIST-1 System\label{sec:trappist}}
% Rory 

\section{Discussion\label{sec:discussion}}

\section{Conclusions\label{sec:conclusions}}

\vspace{1cm}
This work was supported by the NASA Astrobiology Institute's Virtual
Planetary Laboratory under Cooperative Agreement number NNA13AA93A.

\section{Appendix: List of Symbols} % List of symbols: Add yours here!
\begin{multicols}{2}
%
\noindent $a$ = semi-major axis\\
%
$e$ = eccentricity\\
$\varepsilon$ = sign of the phase lag (\eqtide)\\
%
$G$ = Newton's gravitational constant\\
%
$j$ = index\\
%
$k_2$ = Love number of degree 2 (\eqtide)\\
%
$L$ = luminosity\\
%
$M$ = mass\\
$M_*$ = stellar mass\\
$M_p$ = planetary mass\\
$M_j$ = mass of body $j$\\
%
$n$ = mean motion (orbital frequency)\\
%
$\Omega$ = rotational frequency\\
$\Omega_j$ = rotational frequency of body $j$\\
%
$P$ = period\\
$P_{eq}$ = equilibrium spin period (\eqtide)\\
$\psi$ = obliquity\\
$\psi_j$ = obliquity of body $j$\\
%
$Q$ = tidal quality factor (\eqtide)\\
%
$R$ = body radius\\
$r_g$ = radius of gyration\\
%
$T_{eff}$ = effective temperature\\
$t$ = time\\
%
$\xi$ = constant in \eqtide~calculations\\ 
%
$Z$ = constant in \eqtide~calculations\\
\end{multicols}

\bibliography{bib}

\end{document}  