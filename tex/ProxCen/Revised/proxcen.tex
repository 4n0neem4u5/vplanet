\documentclass[preprint,12pt]{aastex}

\usepackage{xspace}
\usepackage{multicol}
\usepackage{color}
\usepackage{rotating}
\usepackage{subfigure}
\usepackage{afterpage}
\usepackage{morefloats}
\def\imagebox#1#2{\vtop to #1{\null\hbox{#2}\vfill}} % http://tex.stackexchange.com/questions/152818/top-aligned-subfigure-with-bottom-aligned-caption
\newcommand{\xxx}[1]{{\color{red} #1}} % Makes things RED
\newcommand{\cn}{\xxx{(cite)}} % CITATION NEEDED
\usepackage[nomarkers,figuresonly]{endfloat} % Push floats to the end

\def\mearth{{\rm\,M_\oplus}}
\def\rearth{{\rm\,R_\oplus}}
\def\msun{{\rm\,M_\odot}}
\def\rsun{{\rm\,R_\odot}}
\def\lsun{{\rm\,L_\odot}}
\def\gsim{~\rlap{$>$}{\lower 1.0ex\hbox{$\sim$}}}
\def\lsim{~\rlap{$<$}{\lower 1.0ex\hbox{$\sim$}}}
\def\etal{{\it et al.\thinspace}}
\def\wpmsq{W m$^{-2}$}
\def\etal{{\it et al.\thinspace}}
\def\eg{{\it e.g.\ }}
\def\etc{{\it etc.\ }}
\def\ie{{\it i.e.\ }}
\def\cf{{\it c.f.\ }}
\def\acen{{$\alpha$~Cen}}

\def\tess{{\it TESS}}
\def\kepler{{\it Kepler}}
\def\jwst{{\it JWST}}

\def\vplanet{\texttt{\footnotesize{VPLANET}}\xspace}
\def\atmesc{\texttt{\footnotesize{ATMESC}}\xspace}
\def\distorb{\texttt{\footnotesize{DISTORB}}\xspace}
\def\distrot{\texttt{\footnotesize{DISTROT}}\xspace}
\def\eqtide{\texttt{\footnotesize{EQTIDE}}\xspace}
\def\poise{\texttt{\footnotesize{POISE}}\xspace}
\def\radheat{\texttt{\footnotesize{RADHEAT}}\xspace}
\def\thermint{\texttt{\footnotesize{THERMINT}}\xspace}
\def\stellar{\texttt{\footnotesize{STELLAR}}\xspace}
\def\galhabit{\texttt{\footnotesize{GALHABIT}}\xspace}

\bibliographystyle{apj}

\usepackage{multicol}
\usepackage{amsmath}
\begin{document}

\title{The Habitability of Proxima Centauri b I: Evolutionary Scenarios}
\author{Rory Barnes\altaffilmark{1,2,3}, Russell Deitrick\altaffilmark{1,2}, Rodrigo Luger\altaffilmark{1,2}, Peter E. Driscoll\altaffilmark{4,2}, Thomas R. Quinn\altaffilmark{1,2}, David P. Fleming\altaffilmark{1,2}, Benjamin Guyer\altaffilmark{1,2}, Diego V. McDonald\altaffilmark{1,2}, Victoria S. Meadows\altaffilmark{1,2}, Giada Arney\altaffilmark{1,2}, David Crisp\altaffilmark{5,2}, Shawn D. Domagal-Goldman\altaffilmark{6,2}, Andrew Lincowski\altaffilmark{1,2}, Jacob Lustig-Yaeger\altaffilmark{1,2}, Eddie Schwieterman\altaffilmark{1,2}}
\altaffiltext{1}{Astronomy Department, University of Washington, Box 951580, Seattle, WA 98195}
\altaffiltext{2}{NASA Astrobiology Institute -- Virtual Planetary Laboratory Lead Team, USA}
\altaffiltext{3}{E-mail: rory@astro.washington.edu}
\altaffiltext{4}{Department of Terrestrial Magnetism, Carnegie Institution for Science, Washington, DC}
\altaffiltext{5}{Jet Propulsion Laboratory, California Institute of Technology, M/S 183-501, 4800 Oak Grove Drive, Pasadena, CA 91109}
\altaffiltext{6}{Planetary Environments Laboratory, NASA Goddard Space Flight Center, 8800 Greenbelt Road, Greenbelt, MD 20771}

\begin{abstract}
We analyze the evolution of the potentially habitable planet Proxima
Centauri b to identify environmental factors that affect its long-term
habitability. We consider physical processes acting on size scales 
ranging between the galactic scale, the scale of the stellar system, 
and the scale of the planet's core.
We find that there is a significant probability that
Proxima Centauri has had encounters with its companion stars, Alpha
Centauri A and B, that are close enough to destabilize Proxima
Centauri's planetary system. If the system has an additional
planet, as suggested by the discovery data, then it may perturb
planet b's eccentricity and inclination, possibly driving those
parameters to non-zero values, even in the presence of strong tidal
damping. We also model the internal evolution of the planet, evaluating
the roles of different radiogenic abundances and tidal heating and
find that a planet with chondritic abundance may not generate a magnetic field, but
all other models do maintain a magnetic field. We
find that if planet b formed {\it in situ}, then it
experienced $169\pm 13$ million years in a runaway greenhouse as the star
contracted during its formation. This early phase may have permanently
desiccated the planet and/or produced a large abiotic
oxygen atmosphere. On the other hand, if Proxima Centauri b formed
with a thin hydrogen atmosphere ($\lesssim~1$\% of the planet's mass), then this envelope could have shielded
the water long enough for it to be retained before being
blown off itself. Through modeling a wide range of Proxima b's evolutionary processes we identify pathways for planet b to be
habitable and conclude that water retention is the biggest obstacle
for planet b's habitability. These results are all obtained with a new
software package called \vplanet.

\end{abstract}

\section{Introduction\label{sec:intro}}

The discovery of Proxima Centauri b, hereafter Proxima b, heralds a
new era in the exploration of exoplanets. Although very little is
currently known about it and its environment, the planet is likely
terrestrial and receives an incident flux that places it in the
``habitable zone'' (HZ)
\citep{Kasting93,Selsis07,Kopparapu13}. Moreover, Proxima b is
distinct from other discoveries in that it is the first potentially
habitable planet that could be directly characterized by space
telescopes such as WFIRST, concept missions such as LUVOIR, HDST, and
HabEx, and/or planned 30-meter class ground-based telescopes. Proxima
b could be the first exoplanet to be spectroscopically probed for
active biology.

The interpretation of these spectra require a firm understanding of
the history of Proxima b and its host system. Proxima b exists in an
environment that is significantly different from Earth and has likely
experienced different phenomena that could preclude or promote the
development of life. When viewed across interstellar distances,
biology is best understood as a planetary process: life is a global
phenomenon that alters geochemical and photochemical
processes. Spectroscopic indicators of life, \ie biosignatures, can
only be identified if the abiotic processes on a planet are understood
-- no single feature in a spectrum is a ``smoking gun'' for life. A
robust detection of extraterrestrial life requires that all plausible
non-biological sources for an observed spectral feature can be ruled
out. This requirement is a tall order in light of the expected
diversity of terrestrial exoplanets in the galaxy and the plethora of
mechanisms capable of mimicking biosignatures
\citep{Schwieterman16,Meadows16}.  With these challenges in mind,
Proxima b may offer the best opportunity to begin the scientific
process of searching for unambiguous signs of life.

In this study, we leverage the known (but sparse) data on Proxima b and
its host system to predict the range of evolutionary pathways that the
planet may have experienced. As we show below, many evolutionary histories are
possible and depend on factors ranging from the cooling rate of b's
core to the orbital evolution of the stellar system through the Milky
Way galaxy, and everything in between. The evolution of Proxima b, and by
extension its potential habitability, depends on physical processes
that tend to be studied by scientists from different
fields, such as geophysics and astrophysics. However, for the purpose
of interpreting Proxima b, we must overcome these divisions. A critical examination of the potential habitability of Proxima b necessitates
a cohesive model that can fold in the impact of the many factors that shape evolutionary
history. Our examination of Proxima b will draw on simple, but realistic, models
that have been developed in the fields of geophysics, planetary
science, atmospheric science and astrophysics. From this synthesis, we
identify numerous opportunities and obstacles for life to develop on
Proxima b, as well as lay a foundation for the future interpretation
of spectroscopic observations.

This paper is organized as follows. In $\S$~\ref{sec:obs} we review
the observational data on the system and the immediate implications
for habitability. In $\S$~\ref{sec:models} we describe models to
simulate the evolution of the system, with a focus on habitability. In
this section we introduce a new software package, \vplanet, which
couples physical models of planetary interiors, atmospheres, spins and
orbits, stellar evolution, and galactic effects. In
$\S$~\ref{sec:results} we present results of these models. An
exhaustive analysis of all histories is too large to present here, so
in this section we present highlights and end-member cases that
bracket the plausible ranges, highlighting some of the issues Proxima b faces with respect to its habitability. In $\S$~\ref{sec:disc} we discuss the
results and identify additional observations that could improve
modeling efforts and connect our results to the companion paper
\citep{Meadows16}. Finally, in $\S$~\ref{sec:concl} we draw our
conclusions.

\section{Observational Constraints \label{sec:obs}}

In this section we review what is known about the triple star system
Alpha Centauri (hereafter \acen) of which Proxima Centauri may be a
third member. This star system has been studied carefully for centuries as
it is the closest to the Sun. We will first review the direct
observational data, then we will make whatever inferences are possible
from those data, and finally we qualitatively consider how these data
constrain the possibility for life to exist on Proxima b, which will
then guide the models described in the next section.

\subsection{Properties of the Proxima Planetary System}
\label{sec:obs:planetsys}
Very little data exist for Proxima b. The radial velocity data
reveal a planet with a minimum mass $m$ of 1.27~$\mearth$, an orbital period $P$
of 11.186 days, and an orbital eccentricity $e$ less than 0.35;
\cite{AngladaEscude16} report a mean longitude $\lambda$ of $110^\circ$. These
data are the extent of the direct observational data on the planet,
but even the minimum mass relies on uncertain estimates of the mass of
the host star, described below.

Proxima b may not be the only planet orbiting Proxima
Centauri. The Doppler data suggest the presence of another planetary
mass companion with an orbital period near 215 days, but it is not
definitive yet \citep{AngladaEscude16}. If present, the second planet has a
projected mass of $\lesssim~10~\mearth$, consistent with previous
non-detections \citep{EndlKurster08,Barnes14,Lurie14}. The orbital
eccentricity and its relative inclination to Proxima b's orbit are also
unknown, but as described below, could take any value that permits
dynamical stability. Additionally, lower mass and/or more distant
planetary companions could also be present in the system.

\subsection{Properties of the Host Star}
\label{sec:obs:star}
As Proxima Centauri is the closest star to the Sun, it has been studied
extensively since its discovery 100 years ago \citep{Innes1915}.  It
has a radius $R_*$ of $0.14~\rsun$, a temperature $T_{eff}$ of 3050 K, a
luminosity $L_*$ of $0.00155~\lsun$~\citep{Boyajian12}, and a rotation
period $P_*$ of 83 days \citep{Benedict98}. \cite{AngladaEscude16} find a 
spectral type of M5.5V. \cite{Wood01} searched for
evidence of stellar winds, but found none, indicating mass loss rates
$\dot{M}_*$ less than 20\% of our Sun's, \ie
$<4~\times~10^{-15}~\msun/\textrm{yr}$. Proxima Centauri possesses a
much larger magnetic field ($B~\sim~600$~G) than our Sun ($B~=~1$~G)
\citep{ReinersBasri08}, but somewhat low compared to the majority of
low mass stars. 

Like our Sun, Proxima Centauri's luminosity varies slowly with time
due to starspots \citep{Benedict93}. HST observations of Proxima
Centauri found variations up to 70 milli-magnitudes (mmag) in $V$
\citep{Benedict98}, which, if indicative of the bolometric luminosity,
corresponds to about a 17.5\% variation, with a
period of 83.5 days (\ie the rotation period). Moreover,
\cite{Benedict98} found evidence for two discrete modes of
variability, one lower amplitude mode ($\Delta~V~\sim~30$~mmag) with a
period of $\sim 42$ days, and a larger amplitude mode
($\Delta~V~\sim~60$~mmag) with a period of 83 days. These periods are
a factor of 2 apart, leading \cite{Benedict98} to suggest that
sometimes a large spot (or cluster of spots) is present on one
hemisphere only, while at other times smaller spots exist on opposite
hemispheres. Regardless, incident stellar radiation (``instellation'')
variations of 17\% could impact atmospheric evolution and surface
conditions of a planet (the sun's variation is of order 0.1\%
\citep{Willson81}).

Additionally, the magnetic field strength may vary with
time. \cite{Cincunegui07} monitored the Ca II H and K lines, which are
indicators of chromospheric activity, as well as H$\alpha$ for 7 years
and found modest evidence for a 442 day cycle in stellar
activity. Although the strength of Proxima's magnetic field at the
orbit of planet b is uncertain, it could affect the stability of
b's atmosphere and potentially affect any putative life on b. 

Proxima Centauri is a known flare star
\citep{Shapley51}\footnote{Although Shapley is the sole author of his
  1951 manuscript, the bulk of the work was performed by two 
  assistants, acknowledged only as Mrs. C.D. Boyd, and Mrs. V.M. Nail.}  and indeed several flares
were reported during the Pale Red Dot campaign
\citep{AngladaEscude16}. \cite{Walker81} performed the first study of
the frequency of flares as a function of energy, finding that high
energy events ($\sim~10^{30}$ erg) occurred about once per day, while
lower energy events ($\sim~10^{28}$ erg) occurred about once per
hour. Numerous observational campaigns since then have continued to
find flaring at about this frequency
\citep{Benedict98,AngladaEscude16,Davenport16}.

\subsection{Properties of the Stellar System}
\label{sec:obs:stellarsys}
Many of the properties of Proxima Centauri are inferred from its 
relationship to \acen~A and B, thus a discussion of the current 
knowledge of \acen is warranted here. 
Proxima Centauri is $\sim 15,000$ AU from \acen~A and B, but all three
have the same motion through the galaxy. The proper motion and radial
velocity of the center of mass of \acen~A and B permit the calculation
of the system's velocity relative to the sun. \cite{Poveda96} find the
three velocities are ($U, V, W$) = (-25, -2, 13) km/s for the center
of mass. This velocity implies the system is currently moving in the
general direction of the Sun, and is on a roughly circular orbit around the galaxy with an
eccentricity of 0.07 \citep{AllenHerrera98}.

A recent, careful analysis of astrometric and HARPS RV data by
\cite{PourbaixBoffin16} found the masses of the two stars to be 1.133
and $0.972~\msun$, respectively, with an orbital eccentricity of 0.52
and a period of 79.91 years. The similarities between both A and B and
the Sun, as well as their low apparent magnitudes, has allowed
detailed studies of their spectral and photometric properties. These
two stars (as well as Proxima) form a foundation in stellar astrophysics,
and hence a great deal is known about A and B. However, as we describe
below, many uncertainties still remain regarding these two stars.

The spectra of \acen~A and B provide information about the stellar
temperature, gravitational acceleration in the photosphere, rotation
rate, and chemical composition. That these features can be
measured turns out to be critical for our analysis of the evolution of
Proxima b. Proxima Centauri is a low mass star with strong molecular
absorption lines and NLTE effects, which make it extraordinarily
difficult to identify elemental abundances; hence its composition
is far more difficult to measure than for G and K dwarfs like \acen~ A
and B \citep{Johnson2009}.  Recently, \cite{HinkelKane13} completed a
reanalysis of published compositional studies, rejecting the studies
of \cite{Laird85} and \cite{NeuforgeMagain97} because they were too
different from the other 5 they considered.  \cite{HinkelKane13}
  found the mean metallicity [Fe/H] of each of the two stars to be
+0.28 and +0.31 and with a large spread of 0.16 and 0.11,
respectively. While it is frustrating that different groups have
arrived at significantly different iron abundances, it is certain the
stars are more metal-rich than the Sun.

\cite{HinkelKane13} go on to examine 21 other elements, including C,
O, Mg, Al, Si, Ca, and Eu. These elements can be important for the
bulk composition and/or are tracers of other species that are relevant
to planetary processes. In nearly all cases, the relative abundance of
these elements to Fe is statistically indistinguishable from the solar
ratios. Exceptions are Na, Zn and Eu in \acen~A, and V, Zn, Ba
and Nd in \acen~B. The discrepancies between the two stars is
somewhat surprising given their likely birth from the same molecular
cloud. On the other hand, the high eccentricity of their orbit could
point toward a capture during the open cluster phase \citep[\eg][]{Malmberg07}. For all
elements beside Eu, the elemental abundances relative to Fe are larger
than in the Sun. In particular, it seems likely that the stars are
significantly enriched in Zn.

\acen~A and B are large and bright enough for asteroseismic
studies that can reveal physical properties and ages of stars to a few
percent, for high enough quality data \citep{Chaplin2014}. Indeed, these two stars
are central to the field of asteroseismology, and have been studied in
exquisite detail \citep[e.g.][]{Bouchy01,Bouchy02}. However, significant uncertainties persist in our
understanding of these stars, despite all the observational
advantages.

A recent study undertook a comprehensive Bayesian analysis of \acen~A
with priors on radius, composition, and mass derived from
interferometric, spectroscopic and astrometric measurements,
respectively \citep{Bazot16}. Their adopted metallicity constraint
comes from \cite{NeuforgeMagain97} via \cite{Thoul03}, which was
rejected by the \cite{HinkelKane13} analysis. They also used an older
mass measurement from \cite{Pourbaix02}, which is slightly smaller
than the updated mass from \cite{PourbaixBoffin16}. They then used an
asteroseismic code to determine the physical characteristics of
A. Although the mass of A is similar to the Sun at $1.1~\msun$, the
simulations of \cite{Bazot16} found that \acen~A's core lies at the
radiative/convective boundary and the transition between pp- and
CNO-dominated energy production chains in the core. Previous results
found the age of \acen~A to be 4.85 Gyr with a convective core
\citep{Thevenin02}, or 6.41 Gyr without a convective core
\citep{Thoul03}. The ambiguity is further increased by uncertainty in
the efficiency of the $^{14}$N(p,$\gamma$)$^{15}$O reaction rate in
the CNO cycle, and by the possibility of
convective overshooting of hydrogen into the core. They also consider
the role of ``microscopic diffusion,'' the settling of heavy
elements over long time intervals. All these uncertainties
prevent a precise and accurate measurement of \acen~A's
age. Combining the different model predictions and including 1$\sigma$
uncertainties, the age of \acen~A is likely to be between 3.4 and 5.9 Gyr,
with a mean of 4.78 Gyr.

\acen~B has also been studied via asteroseismology, but as with A, the
results have not been consistent. \cite{Lundkvist14} find significant
discrepancies between their ``Asteroseismology Made Easy'' age (1.5
Gyr) with other values, but with uncertainties in excess of 4 Gyr. The
asteroseismic oscillations on B are much smaller than on A, which make
analyses more difficult \citep[see, \eg,][]{CarrierBourban03},
leading to the large uncertainty. Combining studies of A and B, we
must conclude that the ages of the two stars are uncertain by at least
25\%. Given the difficulty in measuring B's asteroseismic pulsations,
we will rely on A's asteroseismic data and assume the age of A and B
to be $4.8^{+1.1}_{-1.4}$ Gyr.

\subsection{Inferences from the Observational Data}
\label{sec:obs:inf}
Because Proxima b was discovered indirectly, its properties and
evolution depend critically on our knowledge of the host star's
properties. Although many properties of Proxima Centauri are known, the mass $M_{Prox}$,
age, effective temperature $T$, and composition are not. The spectra and luminosity suggest
the mass of Proxima is $\sim~0.12~\msun$ \citep{Delfosse00}. If we
adopt this value, then the semi-major axis of b's orbit is 0.0485~AU
and the planet receives 65\% of the instellation Earth receives
from the Sun \citep{AngladaEscude16}.  Note that \cite{Sahu14}
suggested that Proxima's proper motion sent it close enough to two
background stars for the general relativistic deflection of their
light by Proxima to be detectable with HST and should allow the determination of
$M_{Prox}$ to better than 10\%, but those results are not yet available.

Additional inferences rely on the assumption that Proxima formed with
the \acen~binary.  The similarities in the proper motion and parallax
between Proxima and \acen~immediately led to speculation as to whether
the stars are ``physically connected or members of the same drift''
\citep{Voute1917}, \ie are they bound or members of a moving group?
The intervening century has failed to resolve this central
question. If Proxima is just a random star in the solar neighborhood,
\cite{MatthewsGilmore93} concluded that the probability that Proxima would
appear so close to \acen~is about 1 in a million, suggesting it is
very likely the stars are somehow associated with each other. Using
updated kinematic information, \cite{Anosova94} concluded that Proxima
is not bound, but instead part of a moving group consisting of about a
dozen stellar systems. \cite{WertheimerLaughlin06}'s reanalysis found
that the observational data favor a configuration that is at the
boundary between bound and unbound orbits. However, their best fit
bound orbit is implausibly large as the semi-major axis is 1.31 pc,
\ie larger than the distance from Earth to
Proxima. \cite{MatvienkoOrlov14} also failed to unequivocally resolve
the issue, and concluded that RV precision of better than 20 m/s is
required to determine if Proxima is bound, which should be available
in the data from \cite{AngladaEscude16}. Perhaps the discovery data
for Proxima b will also resolve this long-standing question.

Regardless of whether Proxima is bound or not, the very low
probability that the stars would be so close to each other strongly
supports the hypothesis that the stars formed in the same star
cluster. We will assume that they are associated and have
approximately equal ages and similar compositions. An age near 5 Gyr
for Proxima is also consistent with its slow rotation period and relatively
modest activity levels and magnetic field \citep{ReinersBasri08}. 

Planet formation around M dwarfs is still relatively understudied, but
it should proceed in a qualitatively similar way as for Sun-like stars,
\ie the planet forms from a disk of dust and gas. Relatively few
observations of disks of M dwarfs exist
\citep[\eg][]{Hernandez07,WilliamsCieza11,Luhman12,Downes15}, but
these data seem to point to a wide range of lifetimes for the gaseous
disks of 1--15~Myr. This timescale is likely longer than the time to form
terrestrial planets in the HZs of late M dwarfs
\citep{Raymond07,Lissauer07}, and hence Proxima b may have been fully
formed before the disk dispersed. For Proxima, the lifetime of the
protoplanetary disk is unknown, and could have been altered by the
presence of \acen~A and B, so any formation pathway or evolutionary
process permitted within this constraint is plausible.

The radial velocity data combined with $M_{Prox}$ only provide a
minimum mass, but significantly larger planet masses are geometrically
unlikely, and very large masses can be excluded because they would
incite detectable astrometric signals (note that the minimum mass
solution predicts an astrometric orbit of the star of $\sim$1
microsecond of arc). It is very likely the planet has a mass less than
10~$\mearth$, and probably $<~3~\mearth$. We will assume the latter
possibility is true, and hence the planet is likely rocky, based on
statistical inferences of the population of \kepler~planets
\citep{WeissMarcy14,Rogers15}. However, even at the minimum mass, we
cannot exclude the possibility that Proxima b possesses a significant
hydrogen envelope, and is better described as a ``mini-Neptune,''
which is unlikely to be habitable \citep[but
  see][]{PierrehumbertGaidos11}.

If non-gaseous, the composition is still highly uncertain and depends
on the unknown formation process. Several possibilities exist
according to recent theoretical studies: 1) the planet formed {\it in
  situ} from local material; 2) the planet formed at a larger
semi-major axis and migrated in while Proxima still possessed a
protoplanetary disk; or 3) an instability in the system occurred that
impulsively changed b's orbit. For case 1, the planet may be depleted
in volatile material \citep{Raymond07,Lissauer07}, but could still
initially possess a significant water reservoir \citep{Ciesla15,Mulders15}. For case 2,
the planet would have likely formed beyond the snow line and hence
could initially be very water-rich \citep{CarterBond12}. For case 3, the planet
could be formed either volatile-rich or poor depending on its initial
formation location as well as the details of the instability, such as
the frequency of impacts that occurred in its aftermath. We conclude
that all options are possible given the data and for simplicity will
assume the planet is Earth-like in composition. If we adopt the
silicate planet scaling law of \cite{Sotin07}, the radius of a
$1.3~\mearth$ planet is $1.07~\rearth$.

\subsection{Implications for Proxima b's Evolution and Habitability}
\label{sec:obs:imp}

Given the above observations and their immediate implications, this
planet may be able to support life. All life on Earth requires three
basic ingredients: Water, energy, and the bioessential elements C, H,
O, N, S and P. Additionally, these ingredients must be present in an
environment that is stable in terms of temperature, pressure and pH
for long periods of time. As we describe in this subsection, these
ingredients may coexist on Proxima b and hence the planet is
potentially habitable, meaning that the planet has long-term persistence
of liquid water on the surface.

Proxima's luminosity and effective temperature combined with b's
orbital semi-major axis place the planet in the HZ of
Proxima and nearly in the same relative position of Earth in the Sun's
HZ in terms of instellation. Specifically, the planet receives about 65\% of Earth's
instellation, which, due to the redder spectrum of Proxima, places b comfortably in the ``conservative'' HZ of 
\cite{Kopparapu13}. Even allowing for observational uncertainties, 
\cite{AngladaEscude16} find that the planet is in this
conservative HZ. 

However, its habitability depends on many more factors than just the
instellation. The planet must form with sufficient water and maintain
it over the course of the system age. Additionally, even if water is present,
the evolution and potential habitability of Proxima b depends on many
other factors involving stellar effects, the planet's internal properties, and the
gravitational influence of the other members of the stellar system.

The host star is about 10 times smaller and less massive than the Sun,
the temperature is about half that of the Sun, and the luminosity is
just 0.1\% that of the Sun. These differences are significant and can
have a profound effect on the evolution of Proxima b. Low mass stars
can take billions of years to begin fusing hydrogen into helium in
their cores, and the star's luminosity can change dramatically during that
time. Specifically, the star contracts at roughly constant temperature and so
the star's luminosity drops with time. For the case of Proxima, this
stage lasted $\sim~1$~Gyr \citep{Baraffe15} and hence Proxima b
could have spent significant time interior to the HZ. This ``pre-main sequence'' (pre-MS) phase could
either strip away a primordial hydrogen atmosphere to reveal a
``habitable evaporated core'' \citep{Luger15}, or, if b formed as a
terrestrial planet with abundant water, it could desiccate that planet during an early runaway greenhouse phase and build up an
oxygen-dominated atmosphere \citep{LugerBarnes15}. Thus, the large
early luminosity of the star could either help or hinder
b's habitability.

Low mass stars also show significant activity, \ie flares, coronal
mass ejections, and bursts of high energy radiation
\citep[\eg][]{West08}. This activity can change the composition of the
atmosphere through photochemistry, or even completely strip the
atmosphere away \citep{Raymond06}. The tight orbit of Proxima b places
it at risk of atmospheric stripping by these phenomena. A planetary
magnetic field could increase the probability of atmospheric retention
by deflecting charged particles, or it could decrease it by funneling
high energy particles into the magnetic poles and providing enough
energy to drive atmospheric escape. Either
way, knowledge of the frequency of flaring and other high energy
events, as well as of the likelihood that Proxima b possesses a
magnetic field, would be invaluable information in assessing the
longevity of Proxima b's atmosphere.

The close-in orbit also introduces the possibility that tidal effects
are significant on the planet. Tides can affect the planet in five
ways. First, they could cause the rotation rate to evolve to a
frequency that is equal to or similar to the orbital frequency, a
process typically called tidal locking
\citep{Dole64,Kasting93,Barnes16}. Second, they can drive the planet's
obliquity $\psi$ to zero or $180^{\circ}$, such that the planet has no seasons
\citep{Heller11}. Third, they can cause the orbital eccentricity to
change, usually (but not always) driving the orbit toward a circular
shape \citep{Darwin1880,FerrazMello08}. Fourth, they can cause
frictional heating in the interior, known as tidal heating
\citep{Peale79,Jackson08c,Barnes13}. Finally, they can cause the
semi-major axis to decay as orbital energy is transformed into
frictional heat, possibly pulling a planet out of the HZ
\citep{Darwin1880,Barnes08}. Except in extreme cases, these processes
are unlikely to sterilize a planet, but they can profoundly affect the
planet's evolution \citep{DriscollBarnes15}.

Many researchers have concluded that tidally locked planets of M
dwarfs are unlikely to support life because their atmospheres would
freeze out on the dark side \citep{Kasting93}. However, numerous
follow-up calculations have shown that tidal locking is not likely to
result in uninhabitable planets
\citep{Joshi97,Pierrehumbert11,Wordsworth11,Yang13,Shields16,Kopparapu16}. These
models all find that winds are able to transport heat to the back side
of the planet for atmospheres larger than about 0.3 bars. In fact,
synchronous rotation may actually allow habitable planets to exist
closer to the host star because cloud coverage develops at the
sub-stellar point and increases the planetary albedo
\citep{Yang13}. Thus, tidal locking may increase a planet's potential
to support life. However, we note that no study has so far considered a tidally locked planet orbiting a star as cool as Proxima.

Although the abundances of elements relative to iron in \acen~A and B,
and by (assumed) extension Proxima, are similar to the Sun's, there is no
guarantee that the abundance pattern is matched in Proxima
b. Planet formation is often a stochastic process and composition
depends on the impact history of a given world. The planet could have
formed near its current location, which would have been relatively hot
early on and the planet could be relatively depleted in volatiles
\citep{Raymond07,Mulders15}. These studies may even overestimate
volatile abundances as they ignored the high luminosities that late M
dwarfs have during planet formation. Alternatively, the planet could
have formed beyond the snow line and migrated in either while the gas
disk was still present, or later during a large scale dynamical
instability. In those cases, the planet could be
volatile-rich.

If the abundances of Proxima are indeed similar to \acen~A and B, then
the depletion of Eu in \acen~A is of note as it is often a tracer
of radioactive material like $^{232}$Th and $^{238}$U
\citep{Young14}. These isotopes are primary drivers of the internal
energy of Earth, and hence if they are depleted in Proxima b, its
internal evolution could be markedly different than Earth's. However,
since no depletion is observed in \acen~B, it is far from clear that
such a depletion exists. One interesting radiogenic possibility is
that the planet could form sufficiently quickly ($\sim~1$~Myr) that
$^{26}$Al could still provide energy to the planet's interior. Hence any
prediction of b's evolution should also consider its role.

The presence of additional planets can change the orbit and obliquity
of planet b through gravitational perturbations. These interactions
can change the orbital angular momentum of b and drive oscillations in
$e$, the inclination $i$, longitude of periastron $\varpi$, and
longitude of ascending node $\Omega$. Changes in inclination can lead
to changes in $\psi$ as the planet's rotational axis is likely fixed
in inertial space, except for precession caused by the stellar torque,
while the orbital plane precesses. These variations can significantly
affect climate evolution and possibly even the planet's potential to
support life \citep{Armstrong14}. 

If Proxima is bound to \acen~A and B, then perturbations by passing
stars and torques by the galactic tide can cause drifts in Proxima's
orbit about A and B \citep{Kaib13}. During epochs of high
eccentricity, Proxima may swoop so close to A and B that their gravity
is able disrupt Proxima's planetary system. This could have occurred
at any time in Proxima's past and can lead to a total rearrangement of
the system. Thus, should additional planets exist in the Proxima
planetary system, these could be present on almost any orbit, with
large eccentricities and large mutual inclinations relative to b's
orbital plane \citep[\eg][]{Barnes11}.

The inferred metallicity of Proxima Centauri is quite high for the
solar neighborhood, which has a mean of -0.11 and standard deviation
of 0.18 \citep{AllendePrieto04}. Indeed, recent simulations of stellar
metallicity distributions in the galaxy find that at the sun's
galactic radius $R_{gal}$ of $\sim$8 kpc, stars cannot form with
[Fe/H] $> +0.15$ \citep{Loebman16}. The discrepancy can be resolved by
invoking radial migration \citep{SellwoodBinney02}, in which stars on
nearly circular orbits are able to ride corotation resonances with
spiral arms either inward and outward. \cite{Loebman16} find that with
migration, the metallicity distribution of stars in the Sloan Digital
Sky Survey III's Apache Point Observatory Galactic Evolution
Experiment \citep{Hayden15} is reproduced. Furthermore, Loebman et
al.\ find that stars in the solar neighborhood with [Fe/H] $> +0.25$
must have formed at $R_{gal} < 4.5$~kpc. Similar conclusions were
reached in an analysis of the RAVE survey by \cite{Kordopatis15}.  We
conclude that this system has migrated outward at least 3.5~kpc, but
probably more. As the surface density scale length of the galaxy is
$\sim$2.5 kpc, this implies that the density of stars at their
formation radius was at least 5 times higher than at the Sun's current
Galactic radius.

The observed and inferred constraints for the evolution of Proxima b
are numerous, and the plausible range of evolutionary pathways is
diverse. The proximity of two solar-type stars complicates the
dynamics, but allows the extension of their properties to Proxima
Centauri. In the next sections we apply quantitative models of the
processes described in this section to the full stellar system in order to
explore the possible histories of Proxima b in detail.

\section{Models\label{sec:models}}

In this section we describe the models we use to consider the
evolution and potential habitability of Proxima b. We generally use
published models that are common to different disciplines of
science. Although the models come from disparate sources, we have
compiled them all into a new software program called \vplanet. This
code is designed to simulate exoplanet evolution, with a focus on
habitability. The problem of habitability is interdisciplinary, but we
find it convenient to break the problem down into more manageable
chunks, which we call ``modules,'' that are incorporated when
applicable. At this time, \vplanet~consists of simple models that are
all representable as sets of ordinary differential equations. Below we
describe qualitatively the modules and direct the reader to the
references for a quantitative description. We then briefly describe
how \vplanet~unifies these modules and integrates the system forward.

\subsection{Stellar Evolution: \stellar}
\label{sec:models:stellar}

Of the many stellar evolutionary tracks available in the literature
\citep[\eg][]{Baraffe98,Dartmouth08,Baraffe15}, we find that the
Yonsei-Yale tracks for low-mass stars \citep{YonseiYale13} provide the
best match to the stellar parameters of Proxima Centauri. We selected
the [Fe/H] = +0.3 track with mixing length parameter
$\alpha_\mathrm{MLT} = 1.0$ and linearly interpolated between the $0.1
\msun$ and $0.15 \msun$ tracks to obtain a track at $M_{Prox} = 0.12
\msun$.  While these choices yield a present-day radius within
$1\sigma$ of $0.1410 \pm 0.0070 \rsun$ \citep{Boyajian12}, the model
predicts a luminosity at $t = 4.78\ \mathrm{Gyr}$ that is $\sim 15\%$
higher than the value reported in \cite{Boyajian12} (a $\sim 10\sigma$
discrepancy). Such a discrepancy is not unexpected, given both the
inaccuracies in the evolutionary models for low mass stars and the
large intrinsic scatter of the luminosity and radius of M dwarfs at
fixed mass and metallicity, likely due to unmodeled magnetic effects
\citep{YonseiYale13}. Moreover, the large uncertainties in the age,
mass, and metallicity of Proxima Centauri (\S\ref{sec:obs}) further
contribute to the inconsistency.

Nevertheless, since we are concerned with the present-day habitability
of Proxima b, it is imperative that our model match the present-day
luminosity of its star. We therefore scale the Yonsei-Yale luminosity
track down to match the observed value, adjusting the evolution of the
effective temperature to be consistent with the radius evolution
(which we do not change). We note that this choice results in a
\emph{lower} luminosity for Proxima Centauri at all ages, which yields
conservative results (``optimistic'' in terms of habitability) for the total amount of water
lost from the planet (\S\ref{sec:results:atmesc}). Moreover, this
adjustment likely has a smaller effect on the qualitative nature of
our results than the large uncertainties on the properties of the star
and the planet.

We also model the evolution of the XUV luminosity of the star as in
\cite{LugerBarnes15}. We use the power-law of \cite{Ribas05} with
power law exponent $\beta = -1.23$, a saturation fraction
$L_\mathrm{XUV}/L_\mathrm{bol} = 10^{-3}$ and a saturation time of 1
Gyr. These choices yield a good match to the present-day value,
$L_\mathrm{XUV}/L_\mathrm{bol} = 2.83\times 10^{-4}$
\citep{Boyajian12}.

\begin{figure}[ht]
\centering
\includegraphics[width=4in]{Figures/placeholder.pdf}
\caption{Luminosity, temperature, radius, and XUV evolution of Proxima
  Centauri from $t_0$ = 1 Myr to the present day. The dashed red lines
  indicate the measured values of each parameter (see text). 1$\sigma$
  uncertainties are shaded in light red. By construction, all tracks
  match the observed values at the present day within 1$\sigma$.}
\label{fig:stellar:evol}
\end{figure}

In Fig.~\ref{fig:stellar:evol} we plot the stellar model used in this
paper, showing the evolution of the luminosity, radius, effective
temperature, and XUV luminosity as a function of time from $t_0$ = 1
Myr to the mean system age of 4.78 Gyr. The long pre-main sequence
(pre-MS) phase studied by \cite{LugerBarnes15} is evident, lasting $\sim$ 
1 Gyr.

\subsection{Atmospheric Escape: \atmesc}
\label{sec:models:atmesc}

We model atmospheric escape under the energy-limited
\citep{Watson81,Erkaev07} and diffusion-limited \citep{Hunten73}
parameterizations, closely following \cite{Luger15} and
\cite{LugerBarnes15}. We refer the reader to those papers for a
detailed description of the equations and methodology. In this section
we outline the main adaptations and improvements to the models
therein.

We model both the escape of hydrogen from a putative thin primordial
envelope and the escape of hydrogen and oxygen from photolysis of
water during an early runaway greenhouse. As in \cite{LugerBarnes15},
we set water escape rates to zero once the planet enters the HZ, since
the establishment of a cold trap should limit the availability of
water in the upper atmosphere. We further assume that planets with
hydrogen envelopes must lose them completely before water can be lost,
given the expected large diffusive separation between light H atoms
and heavy H$_2$O molecules.  We shut off hydrodynamic escape at 1 Gyr,
the approximate time at which the star reaches the main sequence, to
account for the transition to ballistic escape predicted by
\cite{OwenMohanty16}. We assume XUV escape efficiencies
$\epsilon_\mathrm{XUV}$ of 0.15 for hydrogen envelope escape and 0.30
for the escape of a steam atmosphere, whose opacity is larger in the
FUV, leading to additional heating \citep{Sekiya81}. Finally, for
hydrogen-rich planets, we use the radius evolution tracks for
super-Earths of \cite{Lopez12} and \cite{LopezFortney14}, enforcing a
radius of $1.07 \rearth$ when no hydrogen is present.

The rate of escape of a steam atmosphere closely depends on the fate
of photolytically-produced oxygen. We compute the hydrodynamic drag on
oxygen atoms using the formalism of \cite{Hunten87} to obtain oxygen
escape rates, tracking the buildup of O$_2$ in the atmosphere. As in
\cite{Tian15} and \cite{Schaefer16}, we account for the increasing
mixing ratio of O$_2$ at the base of the hydrodynamic flow, which
slows the escape of hydrogen. \cite{Tian15} find that as oxygen
becomes the dominant species in the upper atmosphere, the
\cite{Hunten87} formalism predicts that an oxygen-dominated flow can
rapidly lead to the loss of all O$_2$ from planets around M
dwarfs. However, because of the larger mass of the oxygen atom,
hydrodynamic oxygen-dominated escape requires exospheric temperatures
$\sim m_\mathrm{O}/m_\mathrm{H} = 16$ times higher than that for a
hydrogen-dominated flow, which is probably unrealistic for Proxima
b. Following the prescription of \cite{Schaefer16}, we therefore
shut off oxygen escape once its mixing ratio exceeds $X_\mathrm{O} =
0.6$, switching to the diffusion-limited escape rate of
hydrogen. Finally, as in \cite{LugerBarnes15}, we also consider the
scenario in which sinks at the surface are efficient enough to remove
O$_2$ from the atmosphere at the rate at which it is produced,
resulting in an atmosphere that never builds up substantial amounts of
oxygen. Recently, \cite{Schaefer16} used a magma ocean model to
calculate the rate of O$_2$ absorption by the surface, showing that
final atmospheric O$_2$ pressures may range from zero to hundreds or
even thousands of bars for the hot Earth GJ 1132b. Our two scenarios
(no O$_2$ sinks and efficient O$_2$ sinks) should therefore bracket
the atmospheric evolution of Proxima b.

\subsection{Tidal Evolution: \eqtide}
\label{sec:models:eqtide}
% Rory
To model the tidal evolution of the Proxima system, we will use a
simple, but commonly-used model called the ``constant-phase-lag''
model \citep{Goldreich66,Greenberg09,Heller11}. This model reduces the
evolution to two parameters, the ``tidal quality factor'' $Q$ and the
Love number of degree 2, $k_2$. While this model has known
shortcomings \citep{ToumaWisdom94,EfroimskyMakarov13}, it provides a
qualitatively accurate picture of tidal evolution, and produces
similar results as the competing constant-time lag model
\citep{Heller10,Barnes13,Barnes16}. Moreover, \cite{Kasting93} used
CPL to calculate the ``tidal lock radius.'' For this study, we use the
model described in \cite{Heller11}, and validate it by reproducing the
tidal evolution of the Earth-Moon orbit \citep{MacDonald64} and the
tidal heating of Io \citep{Peale79}.

The values of $Q$ and $k_2$ for Earth are well-constrained by lunar
laser ranging \citep{Dickey94} to be 12 and 0.299, respectively
\citep{Williams78,Yoder95}. However, their values for celestial bodies
are poorly constrained because the timescales for the evolution are
very long. Values for
stars are typically estimated to be of order $10^6$
\citep[\eg][]{Jackson09}; dry terrestrial planets have $Q~\sim~100$
\citep{Yoder95,Henning09}, and gas giants have $Q=10^4-10^6$
\citep{AksnesFranklin01,Jackson08a}. In $\S$~\ref{sec:results} we will
consider the possibility that Proxima b began with a hydrogen envelope
and was perhaps more like Neptune than Earth. There is some debate
regarding the location of tidal dissipation in gaseous exoplanets,
whether it is in the envelope (high $Q$) or in the core (low $Q$)
\citep[\eg][]{StorchLai14}. We will consider planets with very thin
hydrogen envelopes, so we will make this latter assumption and use the $Q$
value computed by \thermint (see $\S$~\ref{sec:models:thermint}) for core-dominated cases.

\subsection{Orbital Evolution: \distorb}
\label{sec:models:distorb}
% Russell
The model for orbital evolution, \distorb (for ``disturbing function orbit evolution''), 
uses the 4th order secular disturbing function from \cite{MurrayDermott99} 
(see their Appendix B), with equations of motion given by Lagrange's planetary 
equations \citep[again, see][]{MurrayDermott99}.
To avoid potential singularities in the equations of motion, we utilize 
the variables
\begin{align}
h & = e \sin{\varpi} \\
k & = e \cos{\varpi} \\
p & = \sin{\frac{i}{2}} \sin{\Omega} \label{eqnp}\\
q & = \sin{\frac{i}{2}} \cos{\Omega} \label{eqnq},
\end{align}
rather than the standard osculating elements $(e,i,\varpi,\Omega)$. This 
variable transformation is straightforward, if tedious, so we do not 
reproduce it here. The resulting form of the disturbing function and 
Lagrange's equations will be explicitly stated in forthcoming works (Barnes
 et al., in prep, Deitrick et al., in prep). Lagrange's equations in this form 
can also be found in \cite{Berger1991}. 

We apply this model to the Proxima b and a possible longer period
companion, hinted at in the RV data. This secular (\ie long-term
averaged) model does not account for the effects of mean-motion
resonances; however, since we apply this to well-separated planets
here, it is adequate for much of our parameter space. Since the model
is 4th order in $e$ and $i$, it can account for coupling of
eccentricity and inclination, although it does begin to break down at
higher eccentricity or mutual inclination. In
Fig.~\ref{fig:orbitvalid} we compare our model to the
{\footnotesize \texttt{HNBody}}\footnote{Publicly available at
 https://janus.astro.umd.edu/HNBody/}
integrator \citep{RauchHamilton02} and find that for modest values of
$e$ and $i$ the two methods are nearly indistinguishable.
 
\begin{figure}
\includegraphics[width=0.5\textwidth]{Figures/placeholder.pdf}
\centering
\caption{Comparison between the \distorb module and the N-body code
  {\footnotesize \texttt{HNBody}} for the high $e,i$ case in Table 1.
  Planet b is represented on the left, putative planet c on the
  right. {\footnotesize \texttt{HNBody}} evolution is in black,
  \distorb in blue for planet b, orange for planet c.}
\label{fig:orbitvalid}
\end{figure} 

\subsection{Rotational Evolution from Orbits and the Stellar Torque: \distrot}
\label{sec:models:distrot}
% Russell
The planetary obliquity is a primary driver of climate, and hence we
also track planet b's evolution carefully. Not only is it responsible
for seasons, but a non-zero obliquity can result in tidal heating
\citep{Heller11}, which can change outgassing rates and atmospheric
properties. Proxima b's obliquity is affected by two key processes:
tidal damping and perturbations from other planets. The \eqtide~module
handles the former, \distrot~the latter.

Our obliquity evolution model, \distrot (for ``disturbing function
rotation evolution''), uses the equations of motion developed in
\cite{Kinoshita1975, Kinoshita1977} and utilized in numerous studies
including \cite{Laskar1986}, \cite{Laskar1993a,Laskar1993b}, and
\cite{Armstrong14}.  It treats the planet as an oblate spheroid
(having an axisymmetric equatorial bulge), with a shape controlled by
the rotation rate (see below). The planet then is subject to a torque
from the host star, which causes axial precession, and changes in its
orbital plane due to perturbations from a companion planet, which
directly change the obliquity angle.  This model is thus dependent on
\distorb through the eccentricity, the inclination, the longitude of
ascending node $(\Omega)$, and the derivatives $dp/dt, dq/dt$
(Eqs.~\ref{eqnp} and \ref{eqnq}).

The equations for obliquity and precession angle $(d\psi/dt,
dp_A/dt)$, also contain a singularity at $\psi = 0$, so we transform
to the variables
\begin{align}
\xi & = \sin{\psi} \sin{p_A} \\
\zeta & = \sin{\psi} \cos{p_A} \\
\chi & = \cos{\psi}.
\end{align}
The third variable, $\chi$, is necessary to preserve domain
information. Hence, we ultimately have three variables to integrate
rather than two.

Since we couple obliquity evolution in \distrot to tidal evolution in
\eqtide, it is necessary to account for changes in the planet's shape
(its dynamical ellipticity) as its rotation rate changes due to
tides. Following the examples of \cite{Atobe2007} and \cite{Brasser2014}, we
scale the planet's oblateness coefficient, $J_2$ (from which the
dynamical ellipticity, $E_d$, can be derived), with the radius $R_p$,
rotation rate $\omega_{rot}$, and mass $M$, as
\begin{equation}
J_2 \propto \frac{\omega_{rot}^2 R_p^3}{M}.
\end{equation}
We use the Earth's $J_2$ as a proportionality factor. As pointed out
by \cite{Brasser2014}, around a rotation period of 13 days, $J_2$
calculated in this way reaches the $J_2$ of Venus, which maintains
this shape at a much slower rotation speed, and so, following their
example, we set the minimum $J_2$ value to the $J_2$ of Venus.
 
In the presence of strong tidal effects, as we would expect at Proxima
b's orbital distance, the obliquity damps extremely quickly (in a few
hundred kyr). However, if another planetary mass companion is present,
then gravitational perturbations can prevent the obliquity from
damping completely. Furthermore, this equilibrium configuration,
called a Cassini state, is confined to a configuration in
which the total angular momentum vector of the planetary system,
$\hat{k}$, the rotational angular momentum vector of the planet,
$\hat{s}$, and the planet's own orbital angular momentum vector,
$\hat{n}$, all lie in the same plane \citep{Colombo1966}.

To identify Cassini states, we use the formula
\begin{equation}
\sin{\Psi} = \frac{(\hat{k}\times \hat{n}) \times (\hat{s} \times \hat{n})}{ | \hat{k}
\times \hat{n}  | \left | \hat{s} \times \hat{n} \right |},
\label{eqn:cassini}
\end{equation}
suggested by \cite[][]{Hamilton2004}.  In a Cassini state, the $\Psi$
will oscillate (with small amplitude) about $0^{\circ}$ or
$180^{\circ}$, so $\sin{\Psi}$ will approach zero.  We will refer to
$\sin{\Psi}$ as the ``Cassini parameter''. If a planet is in a Cassini
state, its obliquity cannot be damped to 0.

\subsection{Radiogenic Heating: \radheat}
\label{sec:models:radheat}

The first of two geophysical modules tracks the abundances of
radioactive elements in the planet's core, mantle and crust. We
consider 5 radioactive species: $^{26}$Al, $^{40}$K, $^{232}$Th,
$^{235}$U, and $^{238}$U. These elements have measured half-lifes of
$7.17 \times 10^5$, $1.251 \times 10^9$, $1.405 \times 10^{10}$,
$7.038 \times 10^8$, and $4.468 \times 10^9$ years, respectively. The
energy associated with each decay is $6.415 \times 10^{-13}$,
$2.134 \times 10^{-13}$, $6.834 \times 10^{-12}$,
$6.555 \times 10^{-12}$ and $8.283 \times 10^{-12}$ J, respectively.

We will consider four different abundance ratios. First, we consider an
Earth-like case with standard abundance concentrations
\citep[\eg][]{Korenaga03,Arevalo09,Huang13}. Note that geoneutrino
experiments are able to measure the decay products of $^{232}$Th and
$^{238}$U \citep{Raghavan98,Araki05,Dye10}.

The second case uses chondritic abundances, in which we augment the
mantle's $^{40}$K budget by a factor of 30 in number to match the
potassium abundance seen in chondritic meteorites
\citep{AndersGrevesse89,Arevalo09}. Such high potassium abundances could be
present if the planet formed beyond the snowline where potassium, a
volatile, is more likely to become embedded in solids.

The third case is a planet containing an initial abundance of 1 part
per trillion (ppt) of $^{26}$Al. If the planet formed within 1 Myr and 
the planetary disk was enriched by a nearby supernova,
either by planetesimal accumulation or a direct collapse in the outer
regions of Proxima's protoplanetary disk, then not all the $^{26}$Al
would have decayed. A planet that formed quickly would likely have
more than 1 ppt of $^{26}$Al, but as we will see in
$\S$~\ref{sec:results}, but this case provides an end-member case for
copmarison. The decay of $^{26}$Al at $t=0$ produces enough heat to melt 1 g
of a CI meteorite, preventing their solidification for several
half-lives \citep{HeveySanders06}. Note that Earth required tens to
hundreds of millions of years to form, so all the primordial $^{26}$Al
in the Solar System had already decayed.

The final case is an inert planet with no radioactive particles. This
case is very unlikely in reality, but serves as a useful end-member
case to bound the evolution of Proxima b.

\subsection{Geophysical Evolution: \thermint}
\label{sec:models:thermint}

We model the coupled core-mantle evolution of Proxima b with a
1-dimensional model that has been calibrated by modern-day Earth
\citep{DriscollBercovici14,DriscollBarnes15}; the
reader is referred to those studies for a comprehensive description. 
Briefly, the model solves for the average core and mantle
temperatures as determined by energy balance in the two layers and 
temperature-dependent parameterizations for heat loss. The code
includes heat transport across the mantle-surface and core-mantle boundaries (CMB),
mantle melt production and eruption
rates, latent heat production by mantle and core solidification, and radiogenic and tidal heating;
see $\S$~\ref{sec:models:eqtide}. Given the thermodynamic state of the core and the pressure of 
the stellar wind at the orbit of Proxima b, a magnetic moment scaling law is used to predict the core 
generated magnetic field and the resulting magnetopause radius.
However, we note that the host star's strong magnetic field may compress the planet's magnetosphere close to 
its surface \citep{Vidotto13,Cohen14}.

Our model has been validated by reproducing the modern Earth's heat budget, 
mantle temperature and eruption flux, inner core radius, and magnetic moment. It
has also been used to produce the divergent evolution
of Venus and Earth under the assumption that they formed with similar compositions 
and temperatures, and that Venus has had a stagnant lid and Earth a mobile 
lid \citep{DriscollBercovici14}. This model does require the calibration of
some uncertain parameters (such as the lower mantle viscosity and core composition), 
and the assumption that Earth and Venus began with the same compositions. While
this model is generic in many ways, it does assume an Earth-like composition, structure, mass and
radius. The minimum mass for Proxima b is close enough to
Earth's for this model to produce first order predictions for its thermal evolution.  We note that
 \thermint is limited to initial mantle temperatures above $\sim$1500~K, below which point
differentiation may not occur, and below 8000~K, where additional phase changes
 would require additional physics.

The \thermint modules can be directly coupled to \eqtide as shown in
\cite{DriscollBarnes15}. In that case, we assume all the tidal power
is deposited in the mantle and the heating changes the temperature, viscosity, and
in turn the tidal $Q$. \cite{DriscollBarnes15} used a visco-elastic
model in which the tidal heating reaches a maximum for mantle
temperature near 1800~K, and thus cooling planets that pass
through this temperature can experience a spike in tidal power
generation.

\subsection{Galactic Effects: \galhabit}
\label{sec:models:galhabit}
% Russell/Tom
Proxima Centauri is tenuously bound, if it is gravitationally bound at all, to 
the binary $\alpha$ Cen A and B. Because of this, it is worthwhile
to investigate the effects of the galactic environment on 
Proxima's orbit. We model the changes produced by galactic tides 
and stellar encounters using the equations and prescriptions 
developed to study the Oort cloud \citep{Heisler1986, Heisler1987, Rickman2008}, as Proxima probably has a similar orbit about $\alpha$ Cen A and B. We utilize an updated galactic 
density of $\rho_0 = 0.102~\msun \rm{pc}^{-3}$ \citep{Holmberg2000} and treat 
$\alpha$ Cen A and B as a single point mass with $M = 2.1$ 
M$_{\odot}$ (with the recently updated masses given by 
\cite{PourbaixBoffin16}). This is a somewhat crude approach, as 
the two stars produce a significant quadrupole moment associated 
with their orbits---a back-of-the-envelope calculation indicates
that the torque associated with this quadrupole potential would 
be equal to the galactic tidal torque at $\sim 2000$ AU. Hence, 
the effect of the binary host should be minor 
at Proxima's estimated distance of $\sim$15,000 AU. The importance is increased if Proxima 
has a significant eccentricity. However, the modeling of the
triple system in a comprehensive way is sufficiently complicated
\citep[see, \eg][]{Harrington1968, Ford2000} to place it beyond 
the scope of this work. Instead, we restrict ourselves to the 
secular effects of galactic tides and passing stars 
and will revisit the triple star dynamics in future work. 

If Proxima is gravitationally bound, galactic tides and stellar 
encounters can pump its eccentricity to values large enough to 
cause disruption from the system, and/or a periastron distance 
so close to the binary $\alpha$ Cen that we would expect 
consequences for any planetary system, such as eccentricity 
excitation. In such situations, Proxima b may have significant tidal 
heating despite the circularization timescale. 

Following \cite{Heisler1987} and \cite{Rickman2008}, we model stellar
encounters with a stochastic Monte Carlo approach, estimating times 
of encounters from the stellar density and velocity dispersion, 
and then randomly drawing stellar magnitudes and velocities 
from the distributions published in \cite{Garciasanchez2001}. 
The impact parameter and velocity are calculated from the relative 
velocities (stellar velocity relative to the apex velocity, 
see \cite{Rickman2008}), and then a $\Delta v$ is applied to
Proxima's orbit according to the impulse approximation
\citep{Remy1985}. The masses of passing stars are calculated 
using the empirical relations from \cite{Reid2002}.

As previously noted, the metallicities of $\alpha$ Cen A and B 
suggest that the system formed at a galactocentric distance of 
$\lesssim 4.5$ kpc \citep{Loebman16}. To model the potential 
effects of radial migration on the triple star system (again, 
assuming Proxima is gravitationally bound to A and B), we 
scale the stellar density and gas density of the galaxy 
according to the radial scale lengths ($R_{\star}, R_{gas}$) found
by \cite{Kordopatis15}. The dark matter density at each distance 
is estimated from their spheroidal model---unlike the disk models 
used for stars and gas, this model is not axisymmetric. However, 
as the dark matter near the mid plane of the disk makes up 
$\lesssim 1\%$ of the total density, it is a decent approximation 
to assume axisymmetry of the total mass density, as the 
\cite{Heisler1986} tidal model assumes. We scale the velocity 
dispersions of the nearby stars as a decaying exponential 
with twice the stellar scale length, $2R_{\star}$, multiplied by 
$\sqrt{t}$, where $t$ is the time since galactic formation, as found 
to be broadly true in galactic simulations 
\citep{Minchev2012, Roskar2012}. In this fashion, the velocity 
dispersion grows slowly in time at all galactic radii, and it grows 
larger closer to the galactic center.
The apex velocity, \ie the velocity of the star with respect to
the Local Standard of Rest, will vary according to the detailed
orbital motion
of Proxima through the galaxy, including the radial migration.  For the
purposes of this study, we
simply keep the apex velocity constant, assuming the current Solar
value is typical, and we will revisit this problem in a later study.

With such scaling laws in place, we model radial migration 
as a single, abrupt jump in the galactocentric distance of the 
system. The reasoning behind this approximation is that N-Body 
simulations show migration to occur generally over the 
course of a single galactic orbit \citep{Roskar2010}; hence, the 
migration time is short compared to the age of the stellar system.
We then randomly choose formation distances over the range 
$(1.5,4.5)$ kpc and migration times over the range $(1,5)$ Gyr 
since formation.

\subsection{The Coupled Model: \vplanet}
\label{sec:models:vplanet}
% Rory
The previously described modules are combined into a single software program
called \vplanet. This code, written in C, is designed to be
modular so that for any given body, only specific modules are applied
and specific parameters integrated in the forward time direction. 
Parameters are integrated
using a 4th order Runge-Kutta scheme with a timestep equal to $\eta$
times the shortest timescale for all active parameters, \ie
$x/(dx/dt)$, where $x$ is a parameter. In general, we obtain convergence if
$\eta~\le~0.01$. A more complete and quantitative description of
\vplanet~will be presented soon (Barnes et al., in prep.). 

Each individual model is validated against observations in our Solar
System or in stellar systems. When possible, conserved quantities are
also tracked and required to remain within acceptable limits. With
these requirements met, we model the evolution of Proxima Centauri b
for plausible formation models to identify plausible evolutionary
scenarios, focusing on cases that allow the planet to be habitable. As
Proxima b is near the inner edge of the HZ, we are primarily concerned
with transitions into or out of a runaway greenhouse. For water-rich
planets, this occurs when the outgoing flux from a planet is $\sim
300$~W/m$^2$ \citep{Kasting93,Abe93} and for dry planets it is at
415~W/m$^2$ \citep{Abe11}. For water-rich planets, we use the
relationship between HZ limits, luminosity and effective temperature
as defined in \cite{Kopparapu13}.

\section{Results\label{sec:results}}
% All

\subsection{Galactic Evolution}
\label{sec:results:galactic}
% Russell

If Proxima is or was bound to \acen~A and B, then Proxima's orbit may
be modified by the galactic tide and perturbations from passing
stars. We ran two experiments to explore the effects of radial
migration: set \textbf{A} places the system in the solar neighborhood,
randomly selecting orbital parameters broadly consistent with the
observed positions, for 10,000 trials. In set \textbf{B}, we have
taken the same initial conditions and randomly selected formation
distances over the range [1.5,4.5] kpc \citep{Loebman16} and migration
times over (1,5) Gyr after formation, after which the system is moved
to the solar neighborhood (8 kpc).

Simulations were halted whenever Proxima's orbit passed within 40 AU
of the center of mass of $\alpha$ Cen A and B, when it passed beyond 1
pc, or when it became gravitationally unbound ($e > 1$). In set
\textbf{A} (see Figure \ref{fig:galacdist}), 1506
out of 10,000 simulations were halted because of one of the three
above conditions.  Closer inspection reveals that the majority of
these (1289) were halted because Proxima's periastron passed within 40
AU of $\alpha$ Cen. Of those that didn't halt, another 1363 passed
within 200 AU and 620 passed within 100 AU. Including those trials
that were halted for any reason, 2688 ($27\%$) passed within 200 AU
and 1929 ($19\%$) passed within 100 AU.

The importance of this distance is that $\sim 100$ AU is the distance a close
encounter with a $\sim 2\msun$ star would disrupt a planetary system
similar to the solar system \citep{Kaib13}. The fact that $\alpha$ Cen
is a binary itself, with a large eccentricity ($e \sim 0.5$) probably
increases the disruption distance still further.  Of course, Proxima
is very different from the Sun and may never have formed a planetary
system like the one we inhabit (with gas giants at large orbital
distances); however, it may still have had an extended planetary
system at some point. If that is the case, the system may have been
disrupted, or may be disrupted in the future, by a close periastron
passage with $\alpha$ Cen. Proxima b may be the remnant of a more
extended planetary system that experienced such a disruption. Or, if 
additional planets remain in the system, \emph{future} close encounters
with \acen~A and B could cause instabilities and chaos within the 
planetary system, so the future evolution should also be considered
for its long-term habitability.

Radial migration, set \textbf{B}, makes close passages and disruption
more likely, as shown in Fig.~ \ref{fig:galacdist}. In this set, 2544
trials were halted and 1717 of those were due to periastron reaching
below 40 AU. Of the remaining cases, 1333 had periastron distances $<
200$ AU and 634 below $<100$ AU. Including cases that were halted,
3195 ($32\%$) passed within 200 AU and 2452 ($25\%$) passed within 100
AU.  An example of the orbital evolution of Proxima with radial
migration is shown in Fig.~\ref{fig:galevolution}.

\begin{figure}
\centering
\includegraphics[width=0.5\textwidth]{Figures/placeholder.pdf}
\caption{Stability of Proxima Cenaturi's orbit without radial
  migration. Distributions of trials in which Proxima's orbit about
  $\alpha$ Cen is disrupted (red) and trials in which it survives to
  greater than the age of the system (blue). {\it Top left:} Initial
  eccentricity. {\it Top right:} Initial semi-major axis. {\it Bottom
    left:} Initial inclination relative to the galactic disk. {\it
    Bottom right:} Minimum periastron distance over the entire
  simulation. Generally, eccentricity and inclination are the greatest
  determinants of stability, with high $e$ and $i \sim 90^{\circ}$
  (\ie low $\hat{Z}$-angular momentum) cases being the least stable.
  Amongst the cases we considered ``stable,'' a significant fraction
  still have Proxima passing within a few hundred AU of $\alpha$ Cen A
  and B.}
\label{fig:galacdist}
\end{figure}

\begin{figure}
\centering
\includegraphics[width=0.5\textwidth]{Figures/placeholder.pdf}
\caption{Same as Fig.~\ref{fig:galacdist} but with radial migration.
  Systems which formed interior to 4.5 kpc from the galactic center
  are disrupted more frequently than
  those which were placed in the solar neighborhood from the
  beginning. }
\label{fig:galacdistmigr}
\end{figure}

% Fig. 1: Sample R_gal vs. time
% Fig. 2: Encounter frequency vs. R_gal
\begin{figure}
\centering
\includegraphics[width=0.5\textwidth]{Figures/placeholder.pdf}
\caption{Stellar encounter rates as a function of galactocentric distance.
  Dark blue points correspond to pre-migration encounter rates, light 
  blue to post-migration. The large outlier in the 
  solar-neighborhood is due to small number statistics---the system 
  in that simulation was disrupted shortly after migration. There is some 
  scatter in the solar-neighborhood points because of the time dependence 
  of the stellar velocity dispersion. At the tail end of the simulations, we 
  match the encounter frequency of 10.5 Myr$^{-1}$ from previous 
  studies \citep{Garciasanchez2001,Rickman2008}.}
\label{fig:encrates}
\end{figure}

% Fig. 3: Sample evolution of Prox's orbit: a) semi, peri, apo, b) incl

\begin{figure}
\centering
\includegraphics[width=0.5\textwidth]{Figures/placeholder.pdf}
\caption{An example of the orbital evolution of Proxima in the galactic 
  simulations. The upper left panel shows the semi-major axis, periastron 
  distance, and apastron distance, the upper right shows the eccentricity, 
  the lower left shows the angular momentum in the $\hat{Z}-$direction, 
  and the lower right shows the inclination with respect to plane of the 
  galactic disk. The system was given a formation distance of $R = 3.78$ 
  kpc and the vertical dashed line shows the time of migration (to 8 kpc). 
  The angular momentum in $\hat{Z}$ (the action $J_z$) is unchanged by 
  galactic tides---eccentricity and inclination exchange angular momentum 
  in such a way that this quantity is conserved---thus its evolution is purely 
  due to stellar encounters. In this particular case, the eccentricity of Proxima 
  grows such that its periastron dips within 50 AU of $\alpha$ Cen A and B.}
\label{fig:galevolution}
\end{figure}

One potential issue with our orbit-averaged approach is that for Proxima in the 
semi-major axis range ($\sim5000$ to $\sim20000$) AU, the orbital periods 
span a range of 240 thousand to 2 million years. Thus it may be possible in 
the simulations for Proxima's periastron to come very close to $\alpha$ Cen 
for a short period of time and then evolve to a larger distance before Proxima 
ever \emph{actually reaches} periastron. Hence, we are potentially halting 
simulations in which 
Proxima may not actually pass within 40 AU of $\alpha$ Cen. Our results 
should thus be seen as an upper limit on the number of configurations 
that lead to close encounters between Proxima and $\alpha$ Cen.
 
\subsection{Orbital/Rotational/Tidal Evolution}
\label{sec:results:orbital}
% Russell/Rory/Diego?

We begin exploring the dynamical properties of the orbits and spins by
considering the tidal evolution of Proxima b if it is in isolation. In
this case, we need only apply \eqtide~to both Proxima and b and track
$a, e, P_{rot},$ and $\psi$. We find that if planet b has $Q=12$,
then an initially Earth-like rotation state becomes tidally locked in
$\sim~10^4$ years, so it seems likely that if b formed near its
current location, then it formed in a tidally locked state and with
negligible obliquity.

Unlike the rotational angular momentum, the orbit can evolve on long
timescales. In the top two panels of Fig.~\ref{fig:eqtide}, we
consider orbits that begin at $a=0.05$~AU and with different
eccentricities of 0.05 (dotted curves), 0.1 (solid curves) and 0.2
(dashed curves). In these cases $a$ and $e$ decrease and the amount of
inward migration depends on the initial eccentricity, which takes 2--3
Gyr to damp to $\sim~0.01$. For initial eccentricities larger than
$\sim~0.23$, the CPL model actually predicts eccentricity growth due
to angular momentum exchange between the star and planet
\citep{Barnes16}. This prediction is likely unphysical and due to the
low order of the CPL model; therefore we do not include evolutionary
tracks for high eccentricities.

The equilibrium tide model posits that the lost rotational and orbital
energy is transformed into frictional heating inside the planet. The
bottom panel of Fig.~\ref{fig:eqtide} shows the average surface energy
flux as a function of time. We address the geophysical implications of
this tidal heating in $\S$~\ref{sec:results:internal:tides}. Note that if planet
b begins with a rotation period of 1 day and an obliquity of
$23.5^\circ$, then the initial surface energy flux due to tidal
heating is $\sim~1$~kW/m$^{2}$.

\begin{figure} 
\begin{center}
\includegraphics[width=0.75\textwidth]{Figures/placeholder.pdf}
\end{center}
\caption{Evolution of planet b's eccentricity (top), semi-major axis
  (middle), and tidal heating surface flux (bottom) assuming that
  initially $a~=~0.05$~AU and $e~=$~0.05 (dotted), 0.1 (solid) or 0.2
  (dashed). For reference the best fit semi-major axis and surface
  energy fluxes of Io and the modern Earth are shown by dashed black
  lines.}
\label{fig:eqtide}
\end{figure}

Next, we consider the role of additional planets, specifically the
putative planet with a 215 day orbit \citep{AngladaEscude16}. For
these runs we now add the \distorb~and \distrot~modules and track the
orbital elements of both planets, the spins of the star and planet b,
and the dynamical ellipticity of planet b. A comprehensive exploration
of parameter space is beyond the scope of this study, so we consider
two end-member cases: a nearly coplanar, nearly circular system, and a
system with high eccentricities and inclinations. The initial orbital
elements and rotational properties of the bodies are listed in Table
\ref{tab:orbitic}.

\begin{table}[h]
\centering
\begin{tabular}{lccccccccc}
\hline\hline \\[-1.5ex]
& $m$ ($M_{\oplus}$)  & $a_s$ (au) & $a_l$ (au) & $e$ & $i$ ($^{\circ}$)
 & $\omega$ ($^{\circ}$) & $\Omega$ ($^{\circ}$) & $\psi$ ($^{\circ}$) & 
 $P_{rot}$ (days)\\[0.5ex]
\hline \\ [-1.5ex]
b & 1.27 & 0.0482817 & 0.05 & 0.001 & 0.001 & 248.87 & 20.68 & 23.5 & 1  \\
c & 3.13 & 0.346 & 0.346 & 0.001 & 0.001 & 336.71 & 20 & &  \\
\hline \\
b & 1.27 & 0.0482817 & 0.05 & 0.02 & 20 & 248.87 & 20.68 & 23.5 & 1  \\
c & 3.13 & 0.346 & 0.346 & 0.02 & 0.001 & 336.71 & 20 & &  \\
\end{tabular}
\caption{Initial conditions for Proxima 2-planet systems. The coplanar, 
  circular case is on top, the eccentric, inclined case below.}
\label{tab:orbitic}
\end{table}

In Fig.~\ref{fig:MultiLow} we show the orbital evolution for the low
$e$ and $i$ case over short (left) and long (right) timescales. As
expected, the planets exchange angular momentum, but over the first
million years there is no apparent drift dueto tidal effects. On
longer timescales, however, we see the eccentricity of $b$ slowly decay
due to tidal heating. Note the differences in timescale for the decay
between Figs.~\ref{fig:eqtide} and \ref{fig:MultiLow}. The
perturbations from a hypothetical ``planet c'' maintain significant
eccentricities for long periods of time.

\begin{figure} 
\begin{center}
\includegraphics[width=0.75\textwidth]{Figures/placeholder.pdf}
\end{center}
\caption{Evolution of orbital elements if a putative planet c exists with an 
orbital period of 215 days and both orbits are nearly circular and nearly 
coplanar. {\it Top Row:} Eccentricity. {\it Bottom Row:} Inclination.}
\label{fig:MultiLow}
\end{figure}

In Fig.~\ref{fig:MultiHigh}, we plot the orbital evolution for the
high $e$ and $i$ case. The eccentricity and inclination oscillations
are longer, and the eccentricity cycles show several frequencies due
to the activation of higher order terms in the coupling coupling of $e$ and
$i$. As in the low $e$ and $i$ case, the eccentricity damps more
slowly than in the unperturbed case. Note as well that the inclination
oscillation amplitude decays with time.

\begin{figure} 
\begin{center}
\includegraphics[width=0.75\textwidth]{Figures/placeholder.pdf}
\end{center}
\caption{Same as Fig.~\ref{fig:MultiLow}, but for the high $e,i$ case.}
\label{fig:MultiHigh}
\end{figure}

In Fig.~\ref{fig:MultiSpins}, we plot the evolution of the rotational
parameters for the two cases. In the top left panel, we show the
evolution of the rotational period. The rotation becomes tidally
locked very quickly (less than 1 Myr for all plausible values of $Q$
for an ocean-bearing world).  In the high $e,i$ case, the planet
briefly enters the 3:2 spin orbit resonance (like the planet Mercury).
The obliquity initially grows due to conservation of angular momentum
\citep{Correia08}, but then damps down. For the high $e,i$ case, the
obliquity reaches an equilibrium value near $0.1^\circ$, while the low
$e,i$ case drops all the way to $10^{-8~\circ}$. The bottom left panel
shows the evolution of the dynamical ellipticity as predicted by the
formulae from \cite{Atobe2007}.  Realistically, the shape of the
planet should lag this shape by a timescale dependent on the planet's
rigidity, but we ignore that delay here. The lower right panel shows the
value of the Cassini Parameter (see Equation \ref{eqn:cassini}) for
the two cases, both of which becomes locked near zero, indicating the
rotational and angular momentum have evolved into a Cassini state (in
this case, Cassini state 2), in which the spin and orbit vectors
of planet b are on opposite sides of the total angular momentum vector
of the planetary system.

\begin{figure} 
\begin{center}
\includegraphics[width=0.75\textwidth]{Figures/placeholder.pdf}
\end{center}
\caption{Evolution of rotational properties of planet b for the
 two hypothetical multiplanet systems, with the low $e,i$ case 
 in blue, and high $e,i$ case in orange. {\it Top left:} Rotation
 Period. {\it Top right:} Obliquity. {\it Bottom left:} Dynamical
 Ellipticity. {\it Bottom right:} Cassini Parameter.}
\label{fig:MultiSpins}
\end{figure}

\subsection{Stellar Evolution}
\label{sec:results:stellar}


In Fig.~\ref{fig:HZEvol} we plot the evolution of the conservative HZ
limits of \cite{Kopparapu13} (blue region) as a function of time; the
HZ is bounded by the runaway greenhouse limit on the side closest to
the star and by the maximum greenhouse limit on the opposite
side. Because of the pre-MS luminosity evolution of
Proxima, the HZ slowly moves inward for $\sim$ 1 Gyr, reaching the
current orbit of Proxima b after $\sim$ 160 Myr.

The figure also shows the ``dry'' HZ limits of \cite{Abe11}, which
apply to planets with very limited surface water ($\lesssim 1\%$ of
the Earth's water inventory); these planets are significantly more
robust to an instellation-triggered runaway. Consequently, if Proxima
b's initial water content was very low, it would have spent
significantly less time in a runaway greenhouse. However, a dry
formation scenario for Proxima b does not help its present-day
habitability. As we show in \S\ref{sec:results:atmesc}, Proxima b
loses 1 ocean of water in $\lesssim 4$ Myr; if it formed with less water
than that, it would be completely desiccated long before entering the
\cite{Abe11} HZ. One can envision cases in which the planet forms dry
but with a protective hydrogen envelope, or forms after $\sim$ 10 Myr,
but such scenarios are unlikely \emph{a priori} and hence we do not consider
them here.

In the atmospheric escape section below, we thus use a value of 160
Myr for the duration of the runaway greenhouse phase on Proxima b,
during which time water is allowed to escape.

\begin{figure}[ht]
\centering
\includegraphics[width=5in]{Figures/HZEvol/hzevol.pdf}
\caption{Evolution of the HZ of Proxima Centauri, along with the orbits of Proxima Centauri b (solid line) 
and Mercury (dashed line). The blue region is the conservative HZ of \cite{Kopparapu13}, while the red lines
are the HZ limits for dry planets of different albedos, $A$, from \cite{Abe11}.}
\label{fig:HZEvol}
\end{figure}

\subsection{Atmospheric Evolution}
\label{sec:results:atmesc}

\xxx{An introduction with discrete scenarios and case study plots, then a 
transition to hard-core MCMC stuff.}

%We consider two broad formation scenarios for Proxima b: one in which
%it formed with abundant water and a thin hydrogen envelope of up to
%1\% by mass (due to either \emph{in situ} accretion or from planetary
%formation farther out followed by rapid disk-driven migration; see
%\cite{Luger15}), and one in which it formed with abundant water but no
%hydrogen. In both cases, we assume a fiducial planet mass of $1.27
%\mearth$.
%
%\begin{figure}[ht]
%\centering
%\includegraphics[width=4in]{Figures/placeholder.pdf}
%\caption{Evolution of the water content and atmospheric O$_2$ pressure
%  on Proxima b for different initial conditions. The initial water
%  content is varied between 1 and 10 TO (various colors) for two
%  different end-member scenarios: no O$_2$ surface sinks (solid lines)
%  and instantaneous oxygen absorption at the surface (dashed
%  lines). The planet mass is held constant at 1.27 M$_\oplus$ and the
%  initial hydrogen envelope fraction is set to zero for all runs. In
%  all but one of the runs, Proxima b is completely desiccated. For an
%  initial water content of 10 TO and no surface sinks, the buildup of
%  $\sim$ 500 bars of atmospheric O$_2$ slows the loss rate of H,
%  preventing the last $\sim$ 1 TO of water from being lost. In the
%  scenario that efficient oxygen sinks are present, the atmospheric
%  O$_2$ mixing ratio never grows sufficiently to limit the escape of
%  H, and desiccation occurs in all cases. Note that in this scenario,
%  the curves in the ``O$_2$ pressure'' panel correspond to the
%  equivalent O$_2$ pressure absorbed at the surface.\vspace{0.2in}}
%\label{fig:atmesc:mirage}
%\end{figure}
%
%\begin{figure}[ht]
%\centering
%\includegraphics[width=4in]{Figures/placeholder.pdf}
%\caption{Evolution of the water and O$_2$ contents assuming Proxima b
%  formed with a hydrogen envelope and 3 TO. Line colors correspond to
%  different initial envelope mass fractions $f_H$, ranging from 0.0001
%  to 0.01. In all cases, the envelope evaporates completely prior to 1
%  Gyr. For $f_H \lesssim 0.001$, the H envelope evaporates quickly
%  enough to allow complete desiccation of the planet prior to its
%  arrival in the HZ at $\sim$ 160 Myr (note the purpleline on the $y$-axis in the top left panel). For $f_H \approx 0.01$, the
%  envelope evaporates a few hundred Myr \emph{after} the planet enters
%  the HZ, preventing any water from being lost and O$_2$ from
%  accumulating in the atmosphere.  This is the most favorable scenario
%  for a potentially habitable Proxima b.\vspace{0.2in}}
%\label{fig:atmesc:hec}
%\end{figure}
%
%In Fig.~\ref{fig:atmesc:mirage} we show the evolution of the latter
%type of planet, which formed with no significant primordial
%envelope. We consider four different initial inventories of water: 1,
%3, 5, and 10 terrestrial oceans (1 TO $\equiv 1.39\times 10^{24}$ g,
%the total mass of surface water on Earth; see \cite{Kasting88}). As
%discussed in $\S$~\ref{sec:models:atmesc}, we also consider two end-member
%scenarios regarding the photolytically-produced O$_2$: no surface
%sinks (solid lines) and efficient surface sinks (dashed lines). In all
%cases but one, the planet is completely desiccated within the first
%160 Myr, building up between tens and hundreds of bars of O$_2$ in
%either its atmosphere or in the solid body. For an initial water
%content of 10 TO and no surface sinks, O$_2$ builds up to high enough
%levels to throttle the supply of H to the upper atmosphere and slow
%the total escape rate. In this scenario, $\sim 1$ TO of water remains,
%alongside a thick 500 bar O$_2$ atmosphere. If Proxima b formed with
%less than ten times Earth's water content, and/or had a persistent
%convecting, reducing magma ocean, it is likely desiccated today.
%
%Next, in Fig.~\ref{fig:atmesc:hec}, we show the results assuming
%Proxima b formed with a hydrogen envelope. We fix the initial water
%content at 3 TO and consider initial envelope mass fractions $f_H$
%ranging from $10^{-4}$ to $10^{-2}$. In all cases, the envelope
%evaporates completely within the first several hundred Myr. For $f_H
%\lesssim 10^{-3}$, the envelope evaporates early enough such that all
%the water is still lost from the planet. For $f_H = 10^{-3}$, only
%about 0.1 TO remain once the planet enters the HZ; only for $f_H \sim
%10^{-2}$ does the presence of the envelope guard against all water
%loss. In these calculations, we assumed inefficient surface sinks, so
%the escape of water at late times was bottlenecked by the presence of
%abundant O$_2$. Planets that form with hydrogen envelopes may have
%quite reducing surfaces, which could absorb most of the O$_2$ and lead
%to even higher total water loss. As before, a few tens to a few
%hundreds of bars of O$_2$ remain in the atmosphere or in the solid
%body at the end of the escape phase.
%
%We note that we obtain slightly more hydrogen loss than
%\cite{OwenMohanty16}, who find that planets more massive than $\sim
%0.9 \mearth$ with $\sim 1\%$ hydrogen envelopes cannot fully lose
%their envelopes around M dwarfs, due primarily to the transition from
%hydrodynamic to ballistic escape at late times. However, their
%calculations were performed for a $0.4 \msun$ M dwarf, whose pre-main
%sequence phase lasts $\sim 200 \mathrm{Myr}$, five times shorter than
%that for Proxima Centauri. Nevertheless, the discrepancy is small: we
%find that for envelope fractions greater than 1\% or masses greater
%than our fiducial value of $1.27 \mearth$, the envelope does not
%completely evaporate, in which case Proxima b would likely be
%uninhabitable.
%
%\begin{figure}[ht]
%\centering
%\includegraphics[width=4in]{Figures/placeholder.pdf}
%\caption{Multiple evolutionary pathways for the water on Proxima
% b. These plots show the final water and final atmospheric O$_2$ content of the
% planet for a suite of different initial conditions, assuming inefficient
% surface sinks for O$_2$. Different marker
% styles indicate different values of the planet mass, the initial
% water content, and the initial hydrogen envelope mass fraction $f_H$
% (the final value of $f_H$ is zero for all planets shown here). 
% Each panel is divided into
% quadrants, corresponding to planets that at the end of the
% simulation have water but no O$_2$ (top left, blue), water and O$_2$ (top
% right, yellow), neither water nor O$_2$ (bottom left, gray), and O$_2$ but no
% water (bottom right, red). Habitable planets are those in region shaded
% blue. Planets in the grey region are desiccated and therefore
% uninhabitable. Planets in the red region are likewise
% uninhabitable, but may have atmospheric O$_2$,
% which could be incorrectly attributed to biology. Finally, planets
% in the yellow region are habitable, since they have abundant surface
% water, but may also have substantial atmospheric O$_2$, which could
% be an impediment to the origin of life. These planets are also
% particularly problematic in the context of atmospheric
% characterization, as the presence of water and O$_2$ could fool
% observers into believing they are inhabited.}
%\label{fig:atmesc:synthA}
%\end{figure}
%
%\begin{figure}[ht]
%\centering
%\includegraphics[width=4in]{Figures/placeholder.pdf}
%\caption{Same as Fig.~\ref{fig:atmesc:synthB}, but assuming Proxima b
% has efficient O$_2$ sinks, preventing the buildup of significant oxygen in the
% atmosphere. The $x$ axis now shows the total amount of oxygen absorbed at the surface.}
%\label{fig:atmesc:synthB}
%\end{figure}
%
%Finally, in Figs.~\ref{fig:atmesc:synthA}--\ref{fig:atmesc:synthB} we present a summary of our
%atmospheric escape calculations, showing the many possible
%evolutionary pathways for Proxima b and how they affect its present
%habitability. The two panels show the final water content (in TO)
%versus the final oxygen abundance (in bars) for different initial
%conditions, assuming inefficient oxygen sinks (Figs.~\ref{fig:atmesc:synthA}) and
%efficient oxygen sinks (Figs.~\ref{fig:atmesc:synthB}). In all cases shown, the final
%hydrogen envelope mass is zero.  Different values of the planet mass,
%initial water content, and hydrogen fraction are indicated with
%different marker styles (see legend at right). Since the axes are
%logarithmic, we appended panels to the left and below the main plot,
%corresponding respectively to final oxygen and water contents of
%zero. In general, habitable outcomes are those that lie in the top
%left of the plot (abundant water, low O$_2$). Planets in the top right
%have abundant water but also build up large amounts of O$_2$, which
%could render them uninhabitable \citep{LugerBarnes15} and/or fool
%observers searching for biosignatures \citep{Schwieterman16}. Planets at the bottom of the
%plot are desiccated and therefore uninhabitable, though some may also
%be biosignature false positives.
%
%While much may be gleaned from the figure, a few results stand
%out. First, a scenario in which Proxima b formed with a hydrogen
%envelope of mass $\sim 0.01 \mearth$ is the most favorable for
%habitability, as such a planet would be a ``habitable evaporated
%core'' \citep{Luger15} and experience no water loss. However, if
%Proxima b is more massive than $1.27 \mearth$, the envelope may not
%completely evaporate and the planet may not be habitable
%today. Second, scenarios in which Proxima b formed with less hydrogen
%can still result in present-day surface water, but in all cases in
%which water remains, $\gtrsim 200$ bars of oxygen are
%retained. Finally, the primary effect of a larger planet mass is to
%increase the amount of oxygen in the atmosphere and/or absorbed at the
%surface; while in some cases a larger planet mass can result in less
%water loss, habitable scenarios are more likely for a lower planet
%mass, consistent with the results of \cite{LugerBarnes15}.
%
%Further data on Proxima b will greatly aid in distinguishing between
%these multiple evolutionary pathways. In particular, constraints on
%the exact mass (via measurements of the orbital inclination) could
%inform the atmospheric escape history, while a value for the radius
%could provide a handle on whether or not a hydrogen envelope is
%present.

We perform a suite of Markov Chain Monte Carlo (MCMC) runs to obtain constraints on the
present-day water content of Proxima Centauri b using the Python code package \texttt{emcee}
\citep{ForemanMackey13}. MCMC allows one to sample from multi-dimensional probability 
distributions that are difficult or impossible to obtain directly, which is the case for
the ensemble of parameters that control the evolution of the planet surface water content
in \texttt{VPLANET}. In this section, we develop a framework for inferring the probability 
distributions of these parameters conditioned on empirical data and our understanding
of the physical processes at play.

The input parameters to our model make up the state vector $\mathbf{x}$:
%
\begin{align}
\label{eq:mcmcx}
\mathbf{x} = \{f_\mathrm{sat}, t_\mathrm{sat}, \beta_\mathrm{xuv}, M_\star, t_\star, a, m\},
\end{align}
%
corresponding, respectively, to the stellar mass, the XUV saturation fraction, the XUV saturation timescale,
the XUV power law exponent, the stellar age, the semi-major axis of the planet, and the
mass of the planet. Given a value of $\mathbf{x}$, \texttt{VPLANET} computes the evolution of the system from
time $t = 0$ to $t = t_\star$, yielding the output vector $\mathbf{y}$:
%
\begin{align}
\label{eq:mcmcy}
\mathbf{y}(\mathbf{x}) = \{L_\star, L_\mathrm{xuv}, t_\mathrm{RG}, m_\mathrm{H}, m_\mathrm{H_2O}, P_\mathrm{O_2}\},
\end{align}
%
corresponding, respectively, to the stellar luminosity, the stellar XUV luminosity, the duration of the 
runaway greenhouse phase, the mass of the
planet's hydrogen envelope, the mass of water remaining on its surface, and the amount of oxygen (expressed
as a partial pressure) retained
in either the atmosphere or the surface/mantle, all of which are evaluated at $t = t_\star$ (i.e., the present day). 
Additional parameters that control the evolution of the 
planet (initial water content, XUV absorption efficiency, etc.) are held fixed in individual runs; see below.

Our goal in this section is to derive posterior distributions for $\mathbf{y}$ (and in particular for $m_\mathrm{H_2O}$) 
given prior information on both 
$\mathbf{x}$ and $\mathbf{y}$. Some parameters---such as the present-day stellar luminosity---are well-constrained,
while others are less well-known and will thus be informed primarily by our choice of prior. This is the case for
the XUV saturation fraction, saturation timescale, and power law exponent, which have been studied in detail 
for solar-like stars \citep{Ribas05} but are poorly constrained for M dwarfs \citep[see, e.g.,][]{LugerBarnes2015}. 
We therefore use flat-log priors for the saturation fraction and timescale, enforcing
$-5 \leq \log(f_\mathrm{sat}) \leq -2$ and $-0.3 \leq \log(t_\mathrm{sat} / \mathrm{Gyr}) \leq 1$. We use
a Gaussian prior for the XUV power law exponent, with a mean of 1.23, the value derived by \citep{Ribas05} for
solar-like stars: $\beta_\mathrm{xuv} \sim \mathcal{N}(-1.23, 0.1)$. We choose an ad hoc standard deviation
$\sigma = 0.1$ and verify \emph{a posteriori} that our results are not sensitive to this choice. As we show
below, $\beta_\mathrm{xuv}$ does not strongly correlate with the total water lost or total
amount of oxygen that builds up on the planet.

We also use a flat prior for the stellar mass ($0.1 \leq M_\star / \mathrm{M}_\oplus \leq 0.15$).
Although stronger constraints on the stellar mass exist \citep[e.g.,][]{Delfosse00,Segransan2003}, these are derived indirectly from mass-luminosity or mass-radius 
relations, which are notoriously uncertain for low mass stars \citep[e.g.,][]{Boyajian12}. We thus
enforce a prior on the present-day luminosity to constrain the value of $M_\star$ via our stellar evolution model (see below).
We enforce a Gaussian prior on the stellar age $t_\star \sim \mathcal{N}(4.8, 1.4^2)$ Gyr based on the constraints discussed
in \S\cn. 

Our prior on the semi-major axis $a$ is a combination of a Gaussian prior on the orbital period, 
$P \sim \mathcal{N}(11.186, 0.002^2)$ days \citep{AngladaEscude16}, and the stellar mass prior. 
Finally, our prior on the planet mass $m$ combines the empirical minimum mass distribution,
$m\sin i \sim \mathcal{N}(1.27, 0.18^2)$ M$_\oplus$ \citep{AngladaEscude16}, and the a priori inclination distribution
for randomly aligned orbits, 
$\sin i \sim \mathcal{U}(0, 1)$, where $\mathcal{U}$ is a uniform distribution \citep[e.g.,][]{Luger17}.

We further condition our model on measured values of the stellar luminosity $L_\star$ and 
stellar XUV luminosity $L_\mathrm{xuv}$. We take
$L_\star \sim \mathcal{N}(1.65, 0.15^2) \times 10^{-3}$ L$_\odot$ \citep{Demory09} and 
$\log L_\mathrm{xuv} \sim \mathcal{N}(-6.36, 0.3^2)$. We base the latter on \cite{Ribas16}, who compiled a comprehensive
list of measurements of the emission of Proxima Centauri in the wavelength range 0.6--118 nm. Summing the fluxes over
this range and neglecting the contribution of flares, we obtain an XUV flux at Proxima Centauri b 
$F_\mathrm{xuv} \approx 252\ \mathrm{erg\ cm^{-2}\ s^{-1}}$, corresponding to $\log L_\mathrm{xuv} = -6.36$ for $a = 0.0485$ AU. Given the
lack of uncertainties for many of the values compiled in \cite{Ribas16} and the fact that some of those estimates
are model extrapolations, it is difficult to establish a reliable error estimate for this value. We make the 
ad hoc but conservative choice $\sigma = 0.3$ dex, noting that the three measurements that inform the X-ray luminosity 
of the star in \cite{Ribas16} (which dominates its XUV emission) have a spread corresponding to $\sigma = 0.2$ dex. 
However, more rigorous constraints on the XUV emission of Proxima Cen with reliable uncertainties are direly needed
to obtain more reliable estimates of water loss from Proxima Cen b.

Given these constraints, we wish to find the posterior distribution of each of the parameters in Equations~(\ref{eq:mcmcx}) 
and~(\ref{eq:mcmcy}). We thus define our likelihood function $\mathcal{L}$ for a given state vector $\mathbf{x}$ as
%
\begin{align}
\ln \mathcal{L}(\mathbf{x}) = &- \frac{1}{2}\left[\frac{(L_\mathrm{\star}(\mathbf{x}) - L_\mathrm{\star})^2}{\sigma_{L_\star}^2}
                              - \frac{(L_\mathrm{xuv}(\mathbf{x}) - L_\mathrm{xuv})^2}{\sigma_{L_\mathrm{xuv}}^2}\right]
                                \nonumber\\
                             &+ \ln \mathrm{Prior}(\mathbf{x}) + C,
\end{align}
%
where $L_\mathrm{\star}(\mathbf{x})$ and $L_\mathrm{xuv}(\mathbf{x})$ are, respectively, the model predictions for the present-day 
stellar luminosity and stellar XUV luminosity given the state vector $\mathbf{x}$, $L_\mathrm{\star}$ and $L_\mathrm{xuv}$ are
their respective observed values, and $\sigma_{L_\star}^2$ and $\sigma_{L_\mathrm{xuv}}^2$ are the uncertainties on those observations. The
$\ln \mathrm{Prior}(\mathbf{x})$ term is the prior probability and $C$ is an arbitrary normalization constant. Expressed in this form,
the observed values of $L_\mathrm{\star}$ and $L_\mathrm{xuv}$ are our ``data,'' while the constraints on the other parameters
are ``priors,'' though the distinction is purely semantic.

\begin{figure}[ht]
 \begin{center}
     \includegraphics[width=0.8\textwidth]{Figures/WaterLoss/star_posteriors.pdf}
      \caption{Posterior distributions for the various stellar parameters used in the model. The first eight
               parameters are model inputs, with their corresponding priors shown in red. The combination of
               these priors and the physical models in \texttt{VPLANET} constrain the stellar and planetary
               parameters shown in this section. Blue curves show Gaussian fits to the posterior distributions,
               with the mean and standard deviation indicated at the top right.
               The last panel shows the duration of the runaway greenhouse
               phase for Proxima Centauri b, one of the model outputs, which we find to be 169 $\pm$ 13 Myr.}
    \label{fig:star_posteriors}
 \end{center}
\end{figure}

Given this likelihood function, we use MCMC to obtain the posterior probability distributions for each of the parameters
of interest. We draw each of the $\mathbf{x}$ from their respective prior distributions and run 40 parallel chains of 5,000 
steps each, discarding the first 500 steps as burn-in. The marginalized posterior distributions for the stellar mass, saturation fraction,
saturation timescale, age, semi-major axis, planet mass, present-day stellar luminosity, present-day stellar XUV luminosity, and 
duration of the runaway greenhouse are shown in Figure~\ref{fig:star_posteriors} as the black histograms. The red curves 
indicate our priors/data, and the purple curve is a Gaussian fit to the runaway greenhouse duration posterior,
yielding $t_\mathrm{RG} = 169 \pm 13$ Myr.

By construction, the planet mass, stellar age, present-day stellar luminosity, and present-day stellar XUV luminosity posteriors reflect
their prior distributions. As mentioned above, the stellar mass posterior is entirely informed by the luminosity posterior via the
\cite{YonseiYale13} stellar evolution tracks. The stellar mass in turn constrains the semi-major axis (via the prior on the period and
Kepler's laws). The XUV saturation fraction is fairly well constrained by the present-day XUV luminosity; a log-normal fit
to its posterior yields $\log\ f_\mathrm{sat} = -3.1 \pm 0.5$, which is fully consistent with the observation that M dwarfs
saturate at or below $\log\ f_\mathrm{sat} \approx -3$ \citep{Jackson2012, Shkolnik2014}. The longer tail at high $f_\mathrm{sat}$ results from the fact that
this parameter is strongly correlated with the saturation timescale, $t_\mathrm{sat}$ (see Figure~\ref{fig:corner} below). If
saturation is short-lived, the initial saturation fraction must be higher to match the present-day XUV luminosity. Interestingly,
our runs do not provide any constraints on $t_\mathrm{sat}$, whose value is equally likely (in log space) across the range 
$[0.5, 10]$ Gyr. Finally, the posterior for the XUV power law exponent $\beta_\mathrm{xuv}$ (not shown in the Figure) is 
the same as the adopted prior, as the present data is insufficient to constrain it.

\begin{figure}[ht]
 \begin{center}
     \includegraphics[width=0.8\textwidth]{Figures/WaterLoss/planet_epsilon.pdf}
      \caption{Marginalized posteriors for the present-day water content (left) and atmospheric oxygen
      pressure (center) on Proxima Cen b. The joint posteriors for these two parameters are shown at the 
      right. \textbf{(a)} Posteriors for the default run ($m_\mathrm{H_2O}^0 = 5$ TO, $m_\mathrm{H}^0 = 0$,
      $\epsilon_\mathrm{xuv} = 0.15$, $\zeta_\mathrm{O_2} = 0$). \textbf{(b)} Same as (a), but for
      $\epsilon_\mathrm{xuv} = 0.05$. \textbf{(c)} Same as (a), but for
      $\epsilon_\mathrm{xuv} = 0.01$. For $\epsilon_\mathrm{xuv} \gtrsim 0.05$, the planet is desiccated or
      almost desiccated and builds up between 500 and 900 bars of O$_2$ in most runs. For $\epsilon_\mathrm{xuv} \sim 0.01$,
      the planet loses less water and builds up less O$_2$, though the loss of more than 1 TO is still likely.
      }
    \label{fig:planet_epsilon}
 \end{center}
\end{figure}

\begin{figure}[ht]
 \begin{center}
     \includegraphics[width=0.8\textwidth]{Figures/WaterLoss/planet_epsilon10.pdf}
      \caption{Similar to Figure~\ref{fig:planet_epsilon}, but for an initial water content
      $m_\mathrm{H_2O}^0 = 10$ TO. As before, the rows correspond to XUV escape efficiencies of
      0.15, 0.05, and 0.01 from top to bottom, respectively. For high XUV efficiency, Proxima
      Cen b loses more than 5 TO in most runs (and is desiccated in $\sim$ 20\% of runs). At lower
      efficiency, the planet loses less water. The amount of O$_2$ that builds up is similar
      to before, but a buildup of more than 1000 bars is now possible.}
    \label{fig:planet_epsilon10}
 \end{center}
\end{figure}

The two quantities that are of the most interest to us --- the final water content $m_\mathrm{H_2O}$ and final O$_2$ atmospheric 
pressure $P_\mathrm{O_2}$ of Proxima Cen b --- depend on four additional parameters we must specify: the initial water content $m_\mathrm{H_2O}^0$, the initial hydrogen 
mass $m_\mathrm{H}^0$ (if the planet formed with a primordial envelope), the XUV escape efficiency $\epsilon_\mathrm{xuv}$, and
the O$_2$ uptake efficiency $\zeta_\mathrm{O_2}$ of the planet surface. In principle, planet formation models could provide
priors on $m_\mathrm{H_2O}^0$ and $m_\mathrm{H}^0$, but such models depend on additional parameters that are unknown or poorly 
constrained. The same is true for the XUV escape efficiency, which can be modeled as in \cite{Ribas16}, and the rate of
absorption of O$_2$ at the surface, which can be computed as in \cite{Schaefer16}. However, given the large number of unknown
parameters needed to constrain these four parameters, for simplicity we perform independent MCMC runs for fixed combinations
of these. By doing this, we circumvent potential biases arising from incorrect priors on these parameters while still
highlighting how our results scale for different assumptions about their values.

In the runs discussed below, our default values are $m_\mathrm{H_2O}^0 = 5$ TO, $m_\mathrm{H}^0 = 0\ \mathrm{M_\oplus}$,
$\epsilon_\mathrm{xuv} = 0.15$, and $\zeta_\mathrm{O_2} = 0$, and we vary each of these parameters in turn. 
Figure~\ref{fig:planet_epsilon} shows the marginalized posterior distributions for the present-day water content (left column)
and present-day O$_2$ atmospheric pressure (middle column), as well as a joint posterior for the two parameters (right column)
for three different values of $\epsilon_\mathrm{xuv}$: \textbf{(a)} 0.15, \textbf{(b)} 0.05, and \textbf{(c)} 0.01. In the first
two cases, the planet loses all or nearly all of the 5 TO it formed with, building up several hundred bars of O$_2$ (with
distributions peaking at about 700 bars and with a spread of several hundred bars). For $\epsilon_\mathrm{xuv} = 0.15$, about
10\% of runs result in no substantial oxygen remaining in the atmosphere; in these runs, the escape was so efficient as to
remove all of the O$_2$ along with the escaping $H$. In the final case, the amount of water lost is significantly smaller:
about 2 TO on average, with a peak in the distribution corresponding to a loss of about 0.8 TO. The amount of O$_2$
remaining is similarly smaller, but still exceeding 100 bars and with similar spread as before. Finally, the joint posterior
plots emphasize how correlated the present-day water and oxygen content of Proxima Cen b are. Since the rate at which
oxygen builds up in the atmosphere is initially constant at first \citep{LugerBarnes15}, and since the amount of water
lost scales with the duration of the escape period, there is a tight linear correlation between the two quantities
(lower right hand corner of the joint posterior plots). However, as the atmospheric mixing ratio of oxygen increases,
the rate at which hydrogen escapes---and thus the rate at which oxygen is produced---begins to decrease, leading to a break
in the linear relationship once $\sim$ 600--700 bars of oxygen build up and leading to the peak in the O$_2$ posteriors
at around that value.

Figure~\ref{fig:planet_epsilon10} is similar to Figure~\ref{fig:planet_epsilon}, but shows runs assuming Proxima Cen b
formed with 10 TO of water. As before, the rows correspond to different escape efficiencies (0.15, 0.05, 0.01, from top
to bottom). The amount of water lost increases in all cases, and for $\epsilon_\mathrm{xuv} = 0.15$ the planet is
desiccated or almost desiccated in about 20\% of runs. The amount of O$_2$ that builds up is similar to that in the
previous figure, but O$_2$ pressures exceeding 1000 bars are now possible in 20--30\% of cases for XUV efficiencies of 0.15
or 0.05.

\begin{figure}[ht]
 \begin{center}
     \includegraphics[width=0.8\textwidth]{Figures/WaterLoss/planet_hydrogen.pdf}
      \caption{Similar to Figure~\ref{fig:planet_epsilon}, but this time varying the initial mass of the primordial
      hydrogen envelope of Proxima Cen b. Other parameters are set to their default values. The initial mass of hydrogen
      is $m_\mathrm{H}^0 =$ \textbf{(a)} $0.01\ \mathrm{M_\oplus}$, \textbf{(a)} $0.001\ \mathrm{M_\oplus}$, and
      \textbf{(a)} $0.0001\ \mathrm{M_\oplus}$. Note the broken axes in the first two rows. In the first two cases,
      no water is lost in more than half of the runs; in such cases, a thin hydrogen envelope remains today. In the
      final case, most planets lost all their hydrogen and all their water. In order to prevent the runaway loss
      of its water, Proxima Cen b must have formed with more than 0.01\% of its mass in the form of a hydrogen envelope.
      }
    \label{fig:planet_hydrogen}
 \end{center}
\end{figure}

In Figure~\ref{fig:planet_hydrogen} we explore the effect of varying the initial hydrogen content of the planet.
From top to bottom, the rows correspond to initial hydrogen masses equal to 0.01, 0.001, and 0.0001~$\mathrm{M_\oplus}$.
In the first two cases, the effect of the envelope is clear, as most planets lose no water and build up no oxygen.
These are mostly cases in which a portion of the hydrogen envelope remains at the present day. However, if the
initial hydrogen mass is on the order of 0.0001~$\mathrm{M_\oplus}$ (corresponding to roughly 100 times Earth's
total atmospheric mass), the shielding effect of the envelope is 
almost negligible; compare panel \textbf{(c)} to the top panel in Figure~\ref{fig:planet_epsilon} (the default run).
In this case, most of the water is lost to space in the majority of the runs.

\begin{figure}[ht]
\begin{center}
    \includegraphics[width=0.8\textwidth]{Figures/WaterLoss/planet_magma.pdf}
     \caption{The same as panel (a) in Figure~\ref{fig:planet_epsilon}, but for efficient surface sinks
     ($\zeta_\mathrm{O_2} = 1$). The O$_2$ posterior now corresponds to the amount of oxygen (in bars) absorbed
     at the planet surface. The absence of atmospheric O$_2$ facilitates the loss of hydrogen, which must
     no longer diffuse through the O$_2$ to escape. In this case, nearly 80\% of runs result in complete desiccation
     (note the broken axis in the first panel). In all cases, Proxima Cen b loses at least 1 TO.}
   \label{fig:planet_magma}
\end{center}
\end{figure}

In Figure~\ref{fig:planet_magma} we show the posteriors assuming the O$_2$ uptake
efficiency of the surface $\zeta_\mathrm{O_2} = 1$, corresponding to instant O$_2$ removal by the surface. Compare
to the top panel of Figure~\ref{fig:planet_epsilon}. In this case, the O$_2$ posterior corresponds to the total amount
of oxygen absorbed by the surface, expressed in bars. While the total amount of oxygen retained by the planet
is similar, the fraction of runs in which the planet loses all of its water increases from $\sim$ 20 to $\sim$ 80.
This occurs because the buildup of atmospheric O$_2$ throttles the escape of hydrogen by decreasing its mixing
ratio in the upper atmosphere; when O$_2$ is quickly absorbed at the surface, hydrogen can escape more easily.

\begin{figure}[hbt]
 \begin{center}
     \includegraphics[width=0.8\textwidth]{Figures/WaterLoss/corner.pdf}
      \caption{Joint posteriors of selected parameters for a run with $\epsilon_\mathrm{xuv} = 0.01$ [same as
       Figure~\ref{fig:planet_epsilon}(c)]. In addition to the correlation between the amount of water
       lost and the amount of O$_2$ that builds up, several strong correlations stand out.
       The strongest ones are between the XUV saturation fraction $f_\mathrm{sat}$ and
       the water content (negative) and O$_2$ pressure (positive). Since most of the water loss occurs 
       in the first few 100 Myr, the value of $f_\mathrm{sat}$ is the single most important parameter
       controlling the present-day water and O$_2$ content of Proxima Cen b. The saturation timescale
       also correlates with the water and oxygen, but not as strongly; for $t_\mathrm{sat} \gtrsim 2$ Gyr,
       its exact value does not significantly affect the evolution of the planet. Shorter saturation
       timescales correlate with higher saturation fractions and therefore indirectly affect the evolution. 
       Interestingly, the correlation between the XUV power law slope $\beta_\mathrm{xuv}$ and the water or
       O$_2$ content is negligible, since once saturation ends the water loss rate plummets --- the final
       water content depends almost entirely on the properties of the star early on.}
    \label{fig:corner}
 \end{center}
\end{figure}

Finally, it is interesting to explore the various correlations between the parameters of the model. It is
clear from the previous figures that the amount of oxygen that builds up strongly correlates with the
amount of water lost from the planet, but additional correlations exist. In Figure~\ref{fig:corner} we plot
the joint posteriors for the XUV saturation fraction, XUV saturation timescale, XUV power law exponent,
present-day XUV luminosity, present-day water content, and present-day O$_2$ content for a run with
$m_\mathrm{H_2O}^0 = 5$ TO, $m_\mathrm{H}^0 = 0$, $\epsilon_\mathrm{xuv} = 0.01$, and $\zeta_\mathrm{O_2} = 0$.
The marginalized posteriors are shown at the top. The strongest correlations are between the final
water and O$_2$ contents and the XUV saturation fraction (first column, bottom two panels). The higher
the XUV saturation fraction, the more water is lost and the more O$_2$ builds up. While this
may be unsurprising, neither the saturation timescale (second column) nor the power law exponent 
(third column) correlate as strongly
with the water and O$_2$ content. For saturation timescales longer than about 2 Gyr, the exact duration
of the saturation phase does not affect the evolution of the planet, since nearly all of the water loss
occurs in the first few 100 Myr. For the same reason, the value of the power law exponent does not
significantly correlate with the water or oxygen. On the other hand, the present-day XUV luminosity does correlate
with water loss, as it implies a higher XUV luminosity at early times.
An accurate determination of $f_\mathrm{sat}$ and more precise measurements of $L_\mathrm{xuv}$ are therefore
critical to determining the evolution of the water content of Proxima Cen b.

\xxx{In summary...}

\subsection{Internal Evolution}
\label{sec:results:internal}

\subsubsection{Role of Radiogenic Abundances}

Modeling the internal evolution of Proxima b is challenging due to
 the large number of unknowns about its composition, 
structure, thermal state,
 atmosphere, and the evolution of its environment, \eg
flares stripping the atmosphere. In this section we first consider how
different abundances of radiogenic isotopes could affect its
evolution, then we consider tidal heating.

As described in $\S$~\ref{sec:models:radheat} we consider four
possible abundance patterns for Proxima b: Earth-like, chondritic, 1
ppt $^{26}$Al, and inert (no radioactivity). In all cases we begin with a core
temperature of 6000~K and a mantle temperature of 3000~K, except for
the chondritic case in which the latter is 4000~K. This change was
necessary 
to avoid numerical issues associated with the high
radiogenic power.

In Fig.~\ref{fig:notides} we show the evolution of the radiogenic
power, mantle temperature, inner core radius, magnetic moment,
magnetopause radius and surface energy flux for the four cases. The
dashed black lines represent the modern Earth's value.

\begin{figure}[ht] 
\begin{center}
\includegraphics[width=0.75\textwidth]{Figures/placeholder.pdf}
\end{center}
\caption{Evolution of internal properties of planet b for different
  assumptions of radiogenic inventory: Earth-like in blue, chondritic
  in orange, 1 part per trillion $^{26}$Al in red, and inert in
  purple. 
  Values for the modern Earth are shown with the dashed black
  line. {\it Top left:} Radiogenic power. The Earth curve is behind
  the $^{26}$Al curve except for time = 0. {\it Top right:} Mantle
  Temperature. In the chondritic case (orange), the mantle heats to
  about 5500~K, where it remains for the next 6~Gyr. For the $^{26}$Al
  (red) case, the kink at 2~Gyr is due to latent heat released as the
  mantle begins to solidify. {\it Middle left}: Size of the solid
  inner core. {\it Middle right}: Magnetic moment. The chondritic case
  has a non-convecting core and no dynamo. {\it Bottom
    left:} Magnetopause radius assuming the solar wind pressure at Proxima b is
  0.2 times that at Earth. {\it Bottom right:} Surface energy
  flux. The hump at 2 Gyr for the $^{26}$Al case is due to mantle
  solidification.}
\label{fig:notides}
\end{figure}

In the top left panel we show the evolution of the total radiogenic
power produced in the core, mantle and surface. Initially, the
power from $^{26}$Al is over $2 \times 10^{18}$~W, but with a
half-life of 700,000 years, its contribution to the energy budget
drops to 0 within $10^7$ years. The Earth-like case is hidden behind
the $^{26}$Al curve except at $t=0$.

The mantle temperature is shown in the top right
panel. The Earth and inert cases are similar, but, as expected, the
inert case temperature drops more quickly with no
radiogenic power in the interior. The enriched cases show interesting
features as the energy transport mechanisms change. The chondritic
case heats to about 5500~K due to the large radiogenic heat
production. The $^{26}$Al mantle solidifies at about 2~Gyr producing
the kink as latent heat is released.

The case with $^{26}$Al is particularly interesting because it
predicts initial mantle temperatures in excess of $10^4$~K, which is
clearly unphysical. While part of the implausible behavior is due to
the ``cold'' initial mantle temperature of 3000~K, the real issue is
that the heating from just 1 ppt of $^{26}$Al is so large that we should expect the silicate mantle to begin to vaporize, which we do not model. However, we do see that after this initial
burst of heating, the planet settles into an Earth-like
evolution. Thus, if heating from $^{26}$Al is just a passing energy
source, it may not affect the evolution. On the other hand, the
heating could be so large that the planet may be irrevocably
altered. At this time, we merely point out that the influence of
$^{26}$Al could be significant for planets with formation times of
order 1~Myr, which is similar to $^{26}$Al's half-life.

In the middle left panel, we show the size of the inner core. The
inert case produces the earliest inner core, though the entire core (which has a radius of 3400~km) does not solidify. The burst of $^{26}$Al
heating causes a slight delay in inner core formation compared to Earth,
while the chondritic case does not experience any core solidification, even
after 7 Gyr.

The middle right panel shows the evolution of the magnetic moment for
the four different cases. The inert and Earth cases are very similar,
but the other two are very different. The $^{26}$Al case has a very
strong magnetic field early on, but it decays rapidly after this power
source is depleted. The chondritic case has no magnetic field because
the core cooling rate is insufficient to drive convection in the
core.

The bottom left panel shows the magnetopause radius for Proxima
assuming a stellar wind pressure that is 20\% weaker than the 
the Earth experiences. 
This value
is an upper limit \citep{Wood01}, so our values here are conservative
--- the actual stand-off radius could be larger --- except for the
chondritic case,
for which there is no magnetic field.

Finally, the bottom right panel shows the surface energy flux for each
case. Not surprisingly the chondritic case maintains the highest heat
fluxes, near 1 W/m$^2$, which is similar to Io's value of 2.5
W/m$^2$~\citep{Veeder94}. The humps and kinks are due to the changing
heat sources described above. The $^{26}$Al case begins with a very
large heat flux, in the range of 100~W/m$^2$, which is nearing the
value to trigger a runaway greenhouse. The inert case emits less
energy than Earth. Note that this lower energy flux partially explains
the similarity in the mantle temperatures with time; the Earth case
produces and loses more energy than the inert case.

\subsubsection{Evolution with Tidal Heating}
\label{sec:results:internal:tides}

Next, we examine the role of tidal heating on the evolution of the planet's interior
and orbit. Rather than also include the different radiogenic
abundances described above, we consider the ``Earth'' case but with
three initial eccentricities: 0.05, 0.1 and 0.2. In
Fig.~\ref{fig:tides} we show the evolution of 9 quantities as a
function of time for the three initial eccentricities.

The thermal evolution of the planet changes significantly
because the tidal power can be in excess of 100 TW, or twice the total
power of the modern Earth. As discussed in \cite{DriscollBarnes15}, we find
that the planet's tidal  
response evolves with the thermal state of the
interior, preventing the unrealistically large tidal power predicted by
simpler equilibrium tide models;
 see above. Essentially, tidal heating 
increases mantle temperature,
lowering mantle viscosity,
which raises the tidal $Q$. In fact, Proxima b
may be close to a
 ``steady-state'' regime as defined in
\cite{DriscollBarnes15},
where the surface heat flow balances tidal dissipation in the mantle so that
planet cools very slowly.  This steady-state relies on the negative feedback between mantle 
temperature and tidal dissipation and the positive feedback between temperature and heat flow, 
so that a decrease in temperature causes increased tidal heating, pushing the temperature back up.

The mantle cooling rate is more sensitive to tidal dissipation than the core because dissipation 
in the model occurs only in the mantle.  The core cooling rate does change somewhat with tidal 
dissipation, but its effect on the magnetic moment is muted because the magnetic moment depends 
on core convective heat flux to the $1/3$ power.
 Note, however, that
the inner core grows earlier and more rapidly than in the
Earth case of Fig.~\ref{fig:notides} because the core is cooling more
rapidly. 
This counterintuitive effect of high tidal heating rates causing rapid cooling and
early core nucleation was also found by \cite{DriscollBarnes15}.
Although none of our cases achieve a fully
solid core, which would quench the dynamo, they are close, and given
the uncertainty in both the composition and structure of Proxima b, 
it is possible that the
core has already solidified, preventing a core dynamo, and exposing the atmosphere to
stellar flares.

\begin{figure}[ht]
\begin{center}
\includegraphics[width=0.5\textwidth]{Figures/placeholder.pdf}
\end{center}
\caption{Evolution of internal properties of planet b for three
  different initial eccentricities, as shown in the legend, and
  assuming the Earth-like levels of radiogenic isotopes. {\it Top
    left:} Power generated by tidal heating. {\it Top middle:} Mantle
  temperature. {\it Top right:} Radius of the inner core. {\it Middle
    left:} Magnetic moment. {\it Middle:} Magnetopause radius. {\it
    Middle right:} Surface energy flux. {\it Bottom left:} Orbital
  eccentricity. {\it Bottom middle:} Tidal $Q$. {\it Bottom right:}
  Semi-major axis.}
\label{fig:tides}
\end{figure}

\subsection{Habitable Evaporated Core Scenarios}

Since a possible path towards habitability for Proxima b is the
``habitable evaporated core'' scenario of \citet{Luger15}, we seek to
model how the presence of an evaporating hydrogen envelope and surface
oceans impact the tidal and orbital evolution of Proxima b.  To do so,
we couple the atmospheric escape physics of \atmesc, tidal evolution
using \eqtide, the Earth-calibrated geophysical interior models of
\radheat and \thermint and the stellar evolution of \stellar.

We model the combined tidal contributions of the envelope, oceans, and
solid interior via the following relation for the body's total tidal
$Q$:
\begin{equation}
\label{eqn:Q_hec}
%-Im(k_2) = -Im(k_2) + \frac{ k_{2_{ocean}}}{Q_{ocean}} +
-Im(k_2) = -Im(k_2)_{interior} + \frac{ k_{2_{ocean}}}{Q_{ocean}} +  % edit by PD
\frac{ k_{2_{envelope}}}{Q_{envelope}},
\end{equation}
where Im($k_2$) is the imaginary part of the Love number \citep[see][]{DriscollBarnes15}. In Eq.~(\ref{eqn:Q_hec}) we remove terms if there is no 
component to contribute to Proxima b's net tidal interaction, \ie no ocean or no envelope.  

In the general case when a hydrogen envelope is present, we only
consider the coupled tidal effects of the interior and the envelope as any
water is likely to be supercritical due to the high pressure exerted
by the envelope.  When an envelope is not present,
we consider the tidal contribution of surface oceans only if the planet is
not in the runaway greenhouse limit since otherwise all water would be present
in the atmosphere.

\begin{figure}[ht]
\centering
\includegraphics[width=1.0\textwidth,height=0.9\textheight]{Figures/placeholder.pdf}
\caption{Evolution of the orbital, tidal and atmospheric properties of
  Proxima b for the ``CPL" case in light blue, ``No Ocean" case in
  red, ``Ocean" case in dark blue, and the ``Envelope" case in orange
  with the dashed line for clarity.  The grey shaded region indicates
  when the planet is in the HZ. {\it Top left:} Surface Flux. {\it Top
    right:} Tidal Q. {\it Middle left:} Orbital Eccentricity. {\it
    Middle right:} Semi-major Axis. {\it Bottom left:} Envelope
  Mass. {\it Bottom right:} Surface Water Mass.}
\label{fig:tidal_hec}
\end{figure}

We simulate four cases that bracket the potential
past tidal evolution of Proxima b.  The first case, ``CPL,'' assumes a
constant tidal $Q = 12$ analogous to the simulations described in
$\S$~ \ref{sec:results:orbital} and consistent with observations of
Earth today \citep{Dickey94,Williams78,Yoder95}.  This low tidal $Q$
is due to efficient energy dissipation by oceans.  The ``No Ocean''
case assumes the tidal interaction is dominated by the planet's
interior as determined by \radheat and \thermint, while the ``Ocean"
case generalizes the ``No Ocean'' case by assuming Proxima b had an
initial inventory of 10 ETO of water with $Q_{ocean} = 12$.  The full
habitable evaporated core ``Envelope'' case considers a planet with a
hydrogen envelope that has an initial mass fraction of $0.001$ of the
planet's total mass with $Q_{envelope} = 10^4$, which is consistent
with measurements of Neptune's tidal Q \citep{ZhangHamilton08}.  The
``Envelope'' case starts with 4.5 TO to demonstrate the envelope's
ability to shield surface water from atmospheric escape.  We set
$k_{2_{ocean}} = 0.3$ and $k_{2_{envelope}} = 0.01$ to let the thermal
interior and oceans determine $k_2$ as these components dominate both
the size and mass of the planet. The results of the simulations are
shown in Figure \ref{fig:tidal_hec}.

The ``No Ocean'' case, dominated by the mantle, reaches tidal a $Q$ of a few
$100$ and undergoes minimal tidally-driven orbital evolution until
after about 1 Gyr.  Initially, the bulk of the surface flux stems from
rotational tidal energy dissipation which lessens as the planet
approaches a tidally locked state at around 15 Myr. Early on in the
``Ocean'' case, the planet is in a runaway greenhouse phase in which all
the water is locked up in the atmosphere and subject to escape of
hydrogen and oxygen from photolysis, decreasing the water mass.  Its
tidal evolution mirrors the ``No Ocean'' case as the mantle dominates.
Once the star's luminosity decreases enough, the planet enters the HZ
at around $10^8$ years and the remaining water condenses to the
surface. The presence of surface oceans dramatically decreases the
tidal $Q$ leading to rapid orbital circularization and a substantial
surface flux increase via tidal energy dissipation.
We note that the tidal $Q$ of the surface oceans may depend on the 
total ocean mass and the presence of shallow seas, which should be explored in 
future studies.

The ``Envelope'' case has a slightly different surface flux and tidal $Q$
than the Ocean and No Ocean cases as the envelope contributes
minimally to the initial tidal evolution and the planet's radius
evolves with time.  Stellar UV flux completely strips the
hydrogen envelope after about $4 \times 10^7$ years, causing the mantle
to again dictate the tidal interaction.  The hydrogen envelope
initially shields the planet's water allowing for enough to survive
subsequent photolysis. About 1 TO of the initial 4.5
remain once the planet enters the HZ if we assume the planet has
inefficient oxygen sinks.  With the envelope gone, surface water
dominates the tides and its evolution mirrors that of the ``Ocean'' case.

\section{Discussion\label{sec:disc}}
% All

In the previous section we outlined numerous possibilities for the
history of Proxima b. We considered processes spanning the planet's
core to the Milky Way galaxy and find that effects at all these scales
could be important in the history of our closest exoplanet. In this
section, we summarize the results in terms of the potential
atmospheres that Proxima b might have, which are considered in
detail in \citep{Meadows16}. Then we examine the likelihood that it is
currently habitable.

\subsection{Atmospheric States}
\label{sec:results:atmstates}

\subsubsection{Earth-Like}
\label{sec:results:atmstates:earthlike}

We find that some scenarios allow for the planet to currently support liquid
surface water. In particular, the ``habitable evaporated core'' scenario
\citep{Luger15} is particularly promising as it can avoid both the
high-luminosity pre-main sequence (pre-MS) phase and any devastating tidal
heating that may occur while the planet ``surfs'' the HZ migration, or
during circularization after orbital destabilization. Such a world may
be enriched in deuterium that remained during hydrogen escape, and it
may also be geothermally active if it is young and enriched in
potassium.

Another possibility is that Proxima b formed at a larger orbital
distance and was scattered in to its current location by a system-wide
instability, which may have been triggered by a close passage with
\acen~A and B, see $\S$~\ref{sec:results:galactic}. If the instability
happened after the star reached the main sequence, then the planet
would have arrived in the HZ after its position had stabilized. If b's
eccentricity after the instability was larger than about 0.35, then
there is some danger that tidal heating could have triggered a runaway
greenhouse (see Fig.~\ref{fig:eqtide}), but it may have been
short-lived enough to permit habitability.

Another low-probability possibility is that even if the planet became
desiccated during the pre-MS phase, impact from water-rich bodies
could simultaneously blow off the CO$_2$ and/or O$_2$ atmosphere while
delivering water. This scenario would require a specific set of events
to occur, but we note that close passages between Proxima and \acen~A
and B could destabilize any putative ``exo-Oort Clouds'' that could
have existed around the stars. Current numerical models do not permit
a robust calculation of this possibility, so it remains a viable path
for Proxima b to be habitable.

Although these atmospheric states permit habitability, we must bear in
mind that the rotation period is either synchronous or in a spin-orbit
resonance due to tidal effects \citep{Rodriguez12}. The rotation rate
is therefore longer than on Earth and so atmospheric dynamics would be
different. If synchronous, it is likely that the planet's daylit side
is covered in clouds \citep{Yang13}.

The obliquity of Proxima b is likely small, even in a Cassini state 
from the influence of companion
planets, due to tidal damping. Thus, we do not expect the 
obliquity to significantly impact the climate in the form of seasons or heat 
transport. So this is one way in which Proxima b's climate will be 
different from Earth. 

\subsubsection{Habitable, but Dry}

As shown in Fig.~\ref{fig:HZEvol}, a dry planet would enter a
potentially habitable state earlier than an Earth-like planet, and
hence it may be that Proxima b is a habitable world, but with little
liquid water. This could occur if the planet formed {\it in situ} and
the gas disk was able to shield the planet from water loss. The
habitable evaporated core scenario is also a possibility if the core
was dry and a thin layer of hydrogen could be blown away. On the other
hand, a planet without much water does not need to spend much time in
a runaway greenhouse to become desiccated. For this latter reason, we
conclude the dry and habitable case to be unlikely, but our models
cannot exclude it.

\subsubsection{Venus-Like}
\label{sec:results:atmstates:venuslike}

Regardless of whether Proxima b spent significant time in a runaway
greenhouse prior to the arrival of the HZ, it could be in a runaway
greenhouse state like Venus. If it formed {\it in situ}, then this
possibility is more likely because its first $\sim$100 Myr were spent
interior to the HZ and hence it may have developed a dense CO$_2$
atmosphere as has occurred on Venus. In this case, the planet is
uninhabitable as the surface temperature is too hot for liquid water,
and the higher surface temperature results from greenhouse warming by
CO$_2$.

Even if the planet avoided desiccation during the pre-MS stage, it is
reasonable to assume that a Venus-like atmosphere is possible. CO$_2$
is a very abundant molecule in planetary atmospheres, and given its
strong ability to heat the surface, high molecular weight, and strong
chemical bonds, it may be able to accumulate in the HZ to large enough
levels to trigger the runaway greenhouse, assuming large enough
reservoirs exist.

A final possibility, mentioned above, is that past tidal heating drove
the planet into a runaway greenhouse \citep{Barnes13}. If the planet
was ever in a high eccentricity state ($e~>~0.35$) then the surface
energy flux from the interior could have reached the critical limit of
$\sim$300 W/m$^2$ \citep{Kasting93,Abe93,Goldblatt15}. Such high
surface fluxes may be short-lived if the heating can only come from
the ocean \citep{DriscollBarnes15}.

\subsubsection{Neptune-Like}
\label{sec:results:atmstates:neptunelike}

Proxima b may have formed with sufficient hydrogen that some has been
retained despite all the high energy processes that can remove
it. This possibility is especially likely if it formed beyond the snow
line and migrated in. Similarly to \cite{OwenMohanty16}, we find that
if Proxima b formed with $\gtrsim$ 1\% of its mass in the form of a
hydrogen envelope, it could still possesses some hydrogen, in which
case the surface may be too hot and/or the surface pressure is too
high for habitability.  Future measurements of Proxima b's radius can
inform its present-day composition and thus settle this issue.
% rodluger: Removed ``perhaps with 10 times more mass than it has now'',
% as that's not really physical. At most it's lost about 1\% H/He.

\subsubsection{Abiotic Oxygen Atmosphere}
\label{sec:results:atmstates:o2atmos}

If Proxima b formed with one or more terrestrial oceans (TO) of water, photolysis followed
by hydrogen escape during the stellar pre-MS phase could have led to
the buildup of substantial O$_2$ in the atmosphere. Although oxygen is
highly reactive, thousands of bars of oxygen can be liberated through
this mechanism \citep{LugerBarnes15} and hence all sinks for it may
become saturated \citep{Schaefer16}. In principle, thousands of bars
of oxygen could remain in the atmosphere, but this figure is most
likely lower, as much of the oxygen will be consumed in the process of
oxidizing the surface.  In this case Proxima b may be uninhabitable,
given that little free energy may be available at the surface for
early organisms to take a hold. Life on Earth is thought to have
emerged in an extremely reducing environment \citep{Oparin24,
  Haldane29}, with access to high energy gradients to fuel early
metabolisms; such a reducing environment may not be present on Proxima b.

In many of the scenarios in which Proxima b develops an O$_2$-rich
atmosphere, it also maintains at least some of its initial water.
After the end of the Pre-MS phase, the O$_2$, H$_2$O and CO$_2$
greenhouse warming could be sufficient to prevent water from
accumulating on the surface, and hence it could have significant
abundance in the stratosphere. While the simultaneous detection of
water and oxygen has traditionally been envisaged as an ideal
combination for life detection \citep{DesMarais2002}, in the case of
Proxima b, it is insufficient to prove habitable conditions exist, let
alone that life is present.

Alternatively, if the greenhouse gases are at low enough levels in the
atmosphere, then it may be possible for liquid water to accumulate on
the planetary surface and this planet would meet the traditional
definition of ``habitable.'' However, as argued above, such a planet
would likely be incapable of supporting life.  Thus, the detection of
atmospheric oxygen as well as the presence of surface liquid by other
means, \eg glint \citep{Robinson10}, would not be sufficient evidence
that the planet is habitable.

\subsubsection{Water and Oxygen, but Uninhabitable?}

Figs.~\ref{fig:atmesc:mirage}---\ref{fig:atmesc:synthB} show an
interesting possibility in which large amounts of oxygen are built up
by the pre-MS runaway, butnot all the water is destroyed. If this
occurs, water and oxygen may both be present in the atmosphere, but
the large O$_2$ inventories may prevent the development of Earth-like
biomolecules. Thus, the simultaneous detection of water and O$_2$ in
Proxima b's atmosphere is not necessarily a biosignature, or really
even habitability in the sense that life could originate.

\subsubsection{No Atmosphere}
\label{sec:results:atmstates:noatmos}

Since Proxima b is subjected to repeated flaring events and other
activity \citep{Walker81,Davenport16}, it may be that the atmosphere
has been permanently destroyed. Such a process is difficult to
envision, as it would require all the volatiles in the mantle to have
been degassed and blown away. However, if the planet was tidally
heated for a long time, mantle convection may have been vigorous and
perhaps total volatile depletion is possible, especially if the planet
is of order 7 Gyr old and if the core has solidified, quenching the
magnetic dynamo \citep{DriscollBarnes15}. Another possibility is that
a recent stellar eruption has temporarily stripped away the
atmosphere, which will reform by outgassing.

\subsection{Is Proxima b Habitable?}
\label{sec:results:habitable}

Planetary habitability is a complicated feature to model
quantitatively and Proxima b is no exception. We do know that the
planet has sufficient energy to support life, assuredly has enough
bioessential elements, and is old enough for life to have gained a
foothold, assuming Earth isn't an extreme outlier. The biggest
questions are if it is terrestrial, and if it possesses vast reservoirs
of liquid water. At this time, it is impossible to determine the
probability that it does support liquid water, so we cannot answer the
eponymous question. As always, more data are needed.

However, our analysis does provide some important information on where
to focus future efforts. As liquid water is vital, it is paramount to
determine the pathways that allow the planet to have accreted and
retained the water. However, even if the planet forms with water, our
investigations have shown that it will not necessarily be retained. If
it formed {\it in situ} or arrived in the HZ at the time of the
dispersal of the gas disk, then Proxima b had to endure $\sim$ 150
Myr in a runaway greenhouse state; see $\S$~\ref{sec:results:stellar} and
\cite{LugerBarnes15}.

Even if the planet arrived in its orbit late, perhaps following an
orbital instability, the water may have to survive a ``tidal
greenhouse'' in which tidal heating drives water loss, see
$\S$~\ref{sec:results:internal} and \cite{Barnes13}. Such high tidal
heating rates may require very large eccentricity and/or abnormally
low $Q$ values, but the former is certainly possible during
planet-planet scattering events \citep{Chatterjee08}, or perhaps by
Kozai-like oscillation driven by perturbations from the \acen~A and B
pair if the orbit of Proxima Centauri was much smaller in the past
\citep{DesideraBarbieri07}. If an additional planet in the system is
massive, on an eccentric orbit, and/or on a highly inclined orbit,
then it, too, may induce perturbations that maintain eccentricities
\citep{TakedaRasio05}, possibly in the range of a tidal greenhouse. A
planet in a Cassini state may receive additional tidal heating,
further increasing the risk of a tidal greenhouse. As shown in
$\S$~\ref{sec:results:internal}, the eccentricity of b will damp, but
the timescale can be very long. Large eccentricities are not
well-modeled by equilibrium tide theory, even with a proper accounting
of geophysical features as in the \thermint~module, so it is difficult
at present to assess the role of tidal heating in water retention.

Another possible route to water loss is through temporary or permanent
atmosphere erosion by flares and coronal mass ejections. It is
possible that these events could blast away the atmosphere completely,
in which case liquid water on the surface is not stable. Should the
atmosphere reform, the water may return to the liquid state, but it is
certainly plausible that some events are powerful enough to remove the
water in one event, or, more likely, repeated bombardments would
slowly remove the atmosphere \citep{Cohen15}.  Our analysis doesn't
provide a direct measurement of this phenomenon, but we note that some
geophysical models for the planet predict high heating rates in the
mantle will suppress core convection. In those cases, the magnetic
field is quenched and is not capable of deflecting charged
particles. Even if the planet does have a magnetic field,
\cite{Vidotto13} find that planets around typical M dwarfs may have
their magnetopause distances driven to the planet surface by the
star's magnetic field. However, it is not clear that a magnetic field
is always beneficial for life, as it also increases the
cross-sectional area of the planet for charged particles and funnels
the energy into the magnetic poles, possibly increasing mass loss.

These processes are all clear dangers for the habitability of Proxima
b. Yet, we are also able to identify pathways that produce decidedly
Earth-like versions of planet b. As shown in Fig.~\ref{fig:tidal_hec},
if the planet formed with 0.1\% of its mass in a hydrogen envelope, 4.5
Earth oceans of water, then the combined effects of the stellar
evolution, envelope evolution and atmospheric escape, tidal evolution,
and geophysical evolution predict a planet with 1 Earth ocean of
surface water, no hydrogen envelope, and a semi-major axis within the
observed uncertainties. A planet like this could evolve similarly to
Earth, and therefore may have Earth-like conditions today.

A final possibility is that Proxima b, receiving only 65\% of Earth's
insolation, may have an ice-covered surface, but with a liquid water
mantle, similar to Europa. For such a planet, the water is heated by
the energy from accretion, radiogenic sources and/or tidal
heating. Ice is much more absorptive at the longer wavelengths of
light that Proxima emits and so it may be difficult to ice over the
planet \citep{JoshiHaberle12,Shields13}, especially since it probably
spent hundreds of Myr in a runaway greenhouse. But if a reflective
haze and/or cloud layer could form, it could reflect away the light
before it reaches the surface \citep[\eg][]{Arney16}. This possibility
seems unlikely, but we cannot rule out that the planet is a
``super-Europa'' \citep{BarnesHeller13}, a scaled up version of the
icy satellites of our Solar System and with a potentially inhabited
subsurface water layer.

Proxima b may or may not be habitable. While we are only able to
identify a narrow range of pathways that permit habitability, we must
bear in mind that our model, while including phenomena over sizescales
of meters to kpc, is simple and does not include many potential
feedbacks. The geochemistry of exoplanets is a gaping hole in
scientific knowledge, and one can easily imagine how other systems may
maintain liquid water with geochemical cycles not present in our Solar
System. Similarly, liquid surface water may represent a sort of
``planetary attractor,'' in which planets in the HZ naturally and
typically evolve toward a state in which their surfaces support liquid
water.

\section{Conclusions\label{sec:concl}}
% Rory
We have performed a detailed analysis of the evolution of the
Alpha Centauri triple star system with a specific focus on Proxima
Centauri b's habitability. We find that many disparate factors are
important, including the stellar system's orbit in the galaxy
($\S$~\ref{sec:results:galactic}), the orbital and rotational
evolution of the planets ($\S$~\ref{sec:results:orbital}), the stellar
evolution ($\S$~\ref{sec:results:stellar}), the geophysical evolution
($\S$~\ref{sec:results:internal}), and the atmospheric evolution
($\S$~\ref{sec:results:atmesc}). We find that many evolutionary
pathways are permitted by the data and hence the planet may currently
exist in one of many possible states.

We conclude that Proxima b may be habitable, but identify the
retention of water as the biggest obstacle for Proxima b to support
life. Water loss may occur through multiple channels operating in
tandem or in isolation, including desiccation during the Pre-MS,
excessive tidal heating, or atmospheric destruction by flares and
coronal mass ejections. We find the most likely pathway for
habitability is if planet b formed with a thin hydrogen envelope of
order $10^{-2}~\mearth$ which was eroded by the early XUV evolution of
the host star; see $\S$~\ref{sec:results:stellar} and
\cite{Luger15}. In that case, Proxima b is a ``habitable evaporated
core'' and has followed a very different trajectory than Earth did on
their paths to liquid surface water.

Regardless of Proxima's habitability, it offers scientists an
unprecedented window into the nature of terrestrial planets. At only
1.3 pc distance, we will be able to study this planet in detail with
future missions, should they be designed appropriately; see
\cite{Meadows16}. If Proxima b is uninhabitable, we may be able to
determine how that happened and how Earth avoided the same fate. At a
minimum, the discovery of Proxima Centauri b has ushered in a new era
in comparative planetology, or perhaps it is the first step in the
discovery of extraterrestrial life.
 
\vspace{1cm} We thank G. Anglada-Escud{\' e} for sharing the results,
and for leading the Pale Red Dot campaing. This work was supported by
the NASA Astrobiology Institute's Virtual Planetary Laboratory under
Cooperative Agreement number NNA13AA93A.  David Fleming is supported
by an NSF IGERT DGE-1258485 fellowship.


\bibliography{bib}

\end{document}  
