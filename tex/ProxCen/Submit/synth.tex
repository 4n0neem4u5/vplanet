\documentclass[preprint,12pt]{aastex}
\usepackage{xspace}
\usepackage{multicol}
\usepackage{color}
\usepackage{rotating}
\usepackage{subfigure}
\def\imagebox#1#2{\vtop to #1{\null\hbox{#2}\vfill}} % http://tex.stackexchange.com/questions/152818/top-aligned-subfigure-with-bottom-aligned-caption
\usepackage{multicol}
\usepackage{amsmath}
\begin{document}

\begin{sidewaysfigure}[h]
\centering
\subfigure{\label{fig:atmesc:synth:a}\imagebox{4in}{\includegraphics[width=5.5in]{synth_a.pdf}}}
\subfigure{\label{fig:atmesc:synth:c}\imagebox{4in}{\includegraphics[width=2.14in]{synth_c.pdf}}}
\caption{Multiple evolutionary pathways for the water on Proxima
  b. These plots show the final water and final atmospheric O$_2$ content of the
  planet for a suite of different initial conditions, assuming inefficient
  surface sinks for O$_2$. Different marker
  styles indicate different values of the planet mass, the initial
  water content, and the initial hydrogen envelope mass fraction $f_H$
  (the final value of $f_H$ is zero for all planets shown here). 
  Each panel is divided into
  quadrants, corresponding to planets that at the end of the
  simulation have water but no O$_2$ (top left, blue), water and O$_2$ (top
  right, yellow), neither water nor O$_2$ (bottom left, gray), and O$_2$ but no
  water (bottom right, red). Habitable planets are those in region shaded
  blue. Planets in the grey region are desiccated and therefore
  uninhabitable. Planets in the red region are likewise
  uninhabitable, but may have atmospheric O$_2$,
  which could be incorrectly attributed to biology. Finally, planets
  in the yellow region are habitable, since they have abundant surface
  water, but may also have substantial atmospheric O$_2$, which could
  be an impediment to the origin of life. These planets are also
  particularly problematic in the context of atmospheric
  characterization, as the presence of water and O$_2$ could fool
  observers into believing they are inhabited.}
\label{fig:atmesc:synthA}
\end{sidewaysfigure}

\begin{sidewaysfigure}[h]
\centering
\subfigure{\label{fig:atmesc:synth:b}\imagebox{4in}{\includegraphics[width=5.5in]{synth_b.pdf}}}
\subfigure{\label{fig:atmesc:synth:c}\imagebox{4in}{\includegraphics[width=2.14in]{synth_c.pdf}}}
\caption{Same as Fig.~\ref{fig:atmesc:synthB}, but assuming Proxima b
  has efficient O$_2$ sinks, preventing the buildup of significant oxygen in the
  atmosphere. The $x$ axis now shows the total amount of oxygen absorbed at the surface.}
\label{fig:atmesc:synthB}
\end{sidewaysfigure}

\end{document}